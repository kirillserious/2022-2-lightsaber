\input{formats/presentation}

\usepackage{amsmath}  % Математические 
\usepackage{amssymb}  % формулы
\usepackage{graphicx}

% Show only current section (without subsection in the frame header)
\setbeamertemplate{headline}{%
  \leavevmode%
  \begin{beamercolorbox}[wd=\paperwidth,ht=2.5ex,dp=1.5ex]{section in head/foot}%
    %\hfill
    \hspace{1em}\strut\insertsectionhead\hspace{.5em}\mbox{}%
  \end{beamercolorbox}%
  %\begin{beamercolorbox}[wd=.5\paperwidth,ht=2.5ex,dp=1.5ex]{subsection in head/foot}%
  %  \mbox{}\hspace{.5em}\strut\insertsubsectionhead\hfill%
  %\end{beamercolorbox}%
}

\title[Магистерская диссертация]
        {Математическое моделирование движений руки, держащей предмет}
\author[К. Ю. Егоров]
        {студент 2 курса магистратуры К. Ю. Егоров\\
        научный руководитель --- к.ф-м.н., доцент И. В. Востриков}
\institute{Кафедра системного анализа\\ ВМК МГУ}
\date{26 апреля 2023}

\begin{document}
    \begin{frame}
        \titlepage
    \end{frame}

    \begin{frame}{Содержание}
        \tableofcontents
    \end{frame}


    \section{Построение математической модели}

    \begin{frame}{Математическое моделирование}
        \centering{
            \hfill
            \includegraphics[width=0.3\textwidth]{img/model/arm.jpeg}
            \hfill
            \includegraphics[width=0.3\textwidth]{img/model/schema.png}
            \hfill
        }
        \begin{block}{Начальные данные}
            \begin{itemize}
                \item Рука --- 3-х сочленённый математический маятник.
                \item Известны длины $\ell_i$, массы $m_i$ и моменты инерции $I_i$.
                \item Фазовые переменные --- углы поворота сочленения $\theta_i$ относительно оси $Ox$. 
            \end{itemize}
        \end{block}
    \end{frame}

    \begin{frame}{Уравнение динамики}
        Метод Эйлера--Лагранжа
        $$
            \mathcal{L} = \Pi - K
            \;\Longrightarrow\;
            \tau_i
            =
              \frac{d}{dt}\left(\frac{\partial \mathcal{L}}{\partial \dot \theta_i}\right)
            - \frac{\partial \mathcal{L}}{\partial \theta_i},
        $$
        где $K$ и $\Pi$ --- общие кинетическая и потенциальная энергии системы.
        \vfill
        \begin{block}{Уравнение динамики}
            $$
                \tau = M(\theta)\ddot\theta + L(\theta, \dot\theta)
            $$
            \begin{itemize}
                \item $\tau_i$~--- момент силы, действующей на $i$-е сочленение
                \item $M(\theta)=M^{\textnormal{T}}(\theta)>0$~--- матрица инерции
                \item $L(\theta, \dot\theta)$~--- вектор центростремительных и корелисовых сил
            \end{itemize}
        \end{block}
    \end{frame}

    \begin{frame}{Математическое моделирование}
        \begin{figure}
            \includegraphics[width=0.6\textwidth]{img/model/pendulum.pdf}
            \caption{Траектория руки в свободном падении}
        \end{figure}
    \end{frame}

    \begin{frame}{Принцип оптимальности}
        Моторно-двигательная система как результат эволюции и обучения строит движение в соотвествии с принципом оптимальности.
        \begin{block}{Биологический принцип оптимальности}
            Выбираемые нервной системой схемы движения являются оптимальными для поставленной задачи.
        \end{block}
        \vfill
        В работе [2] показано, что оптимизации проводится с целью уменьшения затрат энергии.
        \begin{block}{Формализация энергетических затрат [3]}
            $$
                \mbox{Затраты}\, = \int\limits_{t_{start}}^{t_{final}}\|\dot\tau\|^2\,dt
            $$
        \end{block}
    \end{frame}


    \section{Постановка задачи оптимального управления}

    \begin{frame}{Постановка задачи}
        Введем $x = [\theta\;\dot\theta\;\tau]^{\textnormal{T}}$, тогда уравнение динамики примет вид
        $$
            \dot x = A(x) + Bu,\;
            \mbox{где}\,
            A(x) = \left[\begin{aligned}
                x_2 \\
                M^{-1}(x_1)(x_3 - L(x_1, x_2)) \\
                O
            \end{aligned}\right],
            B = \left[\begin{aligned}
                0 \\ 0 \\ I
            \end{aligned}\right].
        $$
        Задано начальное положение:
        $$
            x(t_{\textnormal{start}}) = x^{\textnormal{start}}
        $$
        Задача минимизации функционала:
        $$
            J
            =
            q_{\textnormal{терм.}}(x^{\textnormal{final}})
            +
            w_1\int_{t_{\textnormal{start}}}^{t_{\textnormal{final}}} q_{\textnormal{фаз.}}(x)\,dt
            +
            w_2\int_{t_{\textnormal{start}}}^{t_{\textnormal{final}}} \| u \|^2 \,dt
            \longrightarrow 
            \textnormal{min},
        $$
        где $q_{\textnormal{терм.}}$, $q_{\textnormal{фаз.}}$ выбираются исходя из конкретной постановки задачи.
    \end{frame}

    \begin{frame}{Дискретный вариант задачи}
        Введем сетку по переменной $t = \{t_k\}_{k = 1}^{N+1}$ с шагом $\Delta t$, тогда
        $$
            \left\{
            \begin{aligned}
            &x^{k+1} = f(x^k, u^k) \\
            &x^0 = x^{start},
            \end{aligned}
            \right.
        $$
        где $f(x^k, u^k) = \Delta t [A(x^k) + Bu^k] + x^k$.
        $$
            J
            =
            q^{N+1}(x^{N+1})
            +
            \sum_{k=1}^{N} q^{k}(x^k)
            +
            \sum_{k=1}^{N} r^{k}(u^k)
            \;\longrightarrow\;\textnormal{min},
        $$
        $q^{N+1}(x) = q_{\textnormal{терм.}}(x)$, $q^{k}(x) = w_1 \Delta t q_{\textnormal{фаз.}}(x)$, $r^k(u) = w_2 \Delta t \| u \|^2$.
    \end{frame}


    \section{Алгоритм синтеза управления}

    \begin{frame}{Алгоритм синтеза управления}
        \begin{block}{Идея метода}
            \begin{enumerate}
                \item Пусть на $k$ шаге алгоритма дано \textit{референсное} управление $\bar u$ и соответствующая ему траектория $\bar x$.
                \item Вдоль траектории линеаризуем систему, квадратично аппроксимируем функционал качества.
                \item Построим оптимальную поправку на управление {\centering $\delta u = \delta u(\bar x, \bar u)$ как линейно-квадратичный регулятор аппроксимированной задачи.}
                \item Если поправка выходит за допустимую область, то есть 
                $\xi_1
            \leqslant
            \frac{J(\bar u) - J(\bar u + \delta u^{*}(\eta))}{J_{\delta}(0) - J_{\delta}(\delta u^{*}(\eta))}
            \leqslant
            \xi_2$
                , то берем в качестве поправки $\eta\delta u$.
                \item Если мы не достигли требуемой точности, то есть $|J(\bar u) - J(\bar u + \delta u)| \geqslant \varepsilon$, повторяем действия с новым референсным управлением.
            \end{enumerate}
        \end{block}
    \end{frame}

    \begin{frame}{Аппроксимированная задача}
        Допустим мы имеем некоторое референсное управление~$\bar u$ и соответствующую ему референсную траекторию~$\bar x$.
        \begin{equation}\label{eq:ref-system}
            \left\{\begin{aligned}
                &\delta x^{k+1} = f_x^k \delta x + f_u^k \delta u \\
                &\delta x^0 = 0.
            \end{aligned}\right.
        \end{equation}
        \begin{multline}\label{eq:ref-cost}
            J = q^{N+1} + q_x^{N+1}\tilde x^{N+1} + \frac{1}{2}\langle \tilde x^{N+1}, q_{xx}^{N+1}\tilde x^{N+1} \rangle
            + \\ +
            \sum_{k=1}^{N}\left[ q^{k} + q_x^{k}\tilde x^{k} + \frac{1}{2}\langle \tilde x^{k}, q_{xx}^{k}\tilde x^{k} \rangle \right]
            + \\ +
            \sum_{k=1}^{N}\left[ r^{k} + r_u^{k}\tilde u^{k} + \frac{1}{2}\langle \tilde u^{k}, r_{uu}^{k}\tilde u^{k} \rangle \right],
        \end{multline}
        где $\tilde x^k = \bar x^k + \delta x^k$, $\tilde u^k = \bar u^k + \delta u^k$.
    \end{frame}

    \begin{frame}{Синтез управления}
        \begin{block}{Теорема [Об оптимальной поправке]}
            Оптимальная поправка $\delta u$ для задачи~\eqref{eq:ref-system}-\eqref{eq:ref-cost} вычисляется как
            \begin{equation*}
                \delta u^k = - (r^k_{uu} + (f^k_u)^{\textnormal{T}} S_{k+1} f^k_u)^{-1} ((f^k_u)^{\textnormal{T}} S_{k+1} f^k_u \delta x + v^{k+1} + r^k_u),
            \end{equation*}
            где $S_k$ и $v^{k}$ высчитываются в обратном времени как
            \begin{equation*}
                    S_k = q^k_{xx} + (f^k_x)^{\textnormal{T}} S_{k+1} (I + f^k_u (r^k_{uu})^{-1} (f^k_u)^{\textnormal{T}} S_{k+1})^{-1} f_x^k,
            \end{equation*}
            \begin{multline*}
                v^k = (f^k_x)^{\textnormal{T}} S_{k+1} ( I + f^k_u (r^k_{uu})^{-1} (f^k_u)^{\textnormal{T}} S_{k+1})^{-1}
                \cdot \\ \cdot
                (-f^k_u (r^k_{uu})^{-1} (f^k_u)^{\textnormal{T}} v^{k+1} + f^k_u (r^k_{uu})^{-1} r^k_u) + (f^k_x)^{\textnormal{T}} v^{k+1} + q^k_x
            \end{multline*}
            с граничными условиями
            \begin{equation*}
                    S_{N+1} = q^{N+1}_{xx},
                    \quad
                    v^{N+1} = q^{N+1}_{x}.
            \end{equation*}
        \end{block}
    \end{frame}
    
    \begin{frame}{Пример: Целевое состояние}
        $$
            q_{\textnormal{терм.}}(x) = \| x - x^{\textnormal{final}}\|^2
        $$
        \begin{figure}
            \centering{
                \includegraphics[width=0.45\textwidth]{img/iLQR/pendulum.pdf}
                \includegraphics[width=0.45\textwidth]{img/iLQR/endpoint.pdf}
            }
            \caption{Траектории руки и схвата для задачи при нулевом начальном управлении $\bar u \equiv 0$.}
        \end{figure}
    \end{frame}


    \section{Выбор начальной траектории алгоритма}

    \begin{frame}{Выбор начальной траектории}
        \begin{block}{Мотивация}
            Предложенный метод сходится медленно. Построим начальное референсное управление $\bar u$ так, чтобы:
            \begin{itemize}
                \item Оно удовлетворяло терминальному условию;
                \item Оно \textit{быстро} строилось;
                \item Оно было допустимо для рассматриваемой задачи.
            \end{itemize}
        \end{block}
    \end{frame}

    \begin{frame}{Референсная траектория}
        Приведем систему к линейной, заменой управления
        $$
            v = M^{-1}(x_1)(\tau - L(x_1, x_2))
        $$
        Тогда система примет вид:
        \begin{equation}\label{eq:linear}
            x^{k+1} =  A x^{k} + B v^{k}
        \end{equation}

        Выберем $x^{\textnormal{final}} \in \textnormal{Argmin}_{x} q^{N+1}(x)$ и решим задачу минимизации

        \begin{equation}\label{eq:linear-cost}
            J = \| x-x^{\textnormal{final}} \|^2 + w \sum_{k=1}^{N} \| v^k \|^2 \longrightarrow \textnormal{min}
        \end{equation}

        Получим референсное управление $u$ из соотношения
        $$
            \tau^k = M(x_1^k)v^k_* + L(x_1^k,x_2^k) \;\Longrightarrow\; u^{k} = \frac{\tau^{k} - \tau^{k-1}}{\Delta t}
        $$
    \end{frame}

    \begin{frame}{Референсная траектория}
        Метод динамического программирования даёт решение задачи.
        \vfill
        \begin{block}{Утверждение}
            Оптимальное управление $v_*$ задачи \eqref{eq:linear}-\eqref{eq:linear-cost} высчитывается как
            $$
                v^k_* = -[wI + B^{\textnormal{T}}P^kB]^{-1} P^k B A x,
            $$
            где матрица $P^k$ может быть посчитана в обратном времени как решение уравнения Риккати
            $$
                \begin{aligned}
                    &P^{k-1} = A^{\textnormal{T}} P^{k} A - A^{\textnormal{T}} P^k B [wI + B^{\textnormal{T}} P^k B]^{-1} B^{\textnormal{T}} P^{k} A
                    \\
                    &P^{N+1} = I
                \end{aligned}
            $$
        \end{block}
    \end{frame}

    \begin{frame}{Пример: целевое состояние}
        $$
            q_{\textnormal{терм.}}(x) = \| x - x^{\textnormal{final}}\|^2
        $$
        \begin{figure}
            \centering{
                \includegraphics[width=0.45\textwidth]{img/initial/pendulum.pdf}
                \includegraphics[width=0.45\textwidth]{img/initial/endpoint.pdf}
            }
            \caption{Траектории руки и схвата для задачи при новом начальном управлении.}
        \end{figure}
    \end{frame}

    \begin{frame}{Пример: целевое состояние}
        $$
            q_{\textnormal{терм.}}(x) = \| x - x^{\textnormal{final}}\|^2
        $$
        \begin{figure}
            \centering{
                \includegraphics[width=0.6\textwidth]{img/initial/compare}
            }
            \caption{Сравнение скорости сходимости алгоритма в зависимости от начальной траектории.}
        \end{figure}
    \end{frame}


    \section{Применение в классических задачах}

    \begin{frame}{Задача: целевое положение схвата}
        $$
            q_{\textnormal{терм.}}(x) = \| e_{\textnormal{схват}}(x) - e^{\textnormal{final}}\|^2
        $$
        \begin{figure}
            \centering{
                \includegraphics[width=0.45\textwidth]{img/examples/reaching_pendulum}
                \includegraphics[width=0.45\textwidth]{img/examples/reaching_endpoint}
            }
            \caption{Траектории руки и схвата для задачи.}
        \end{figure}
    \end{frame}

    \begin{frame}{Задача: целевое положение и скорость схвата}
        $$
            q_{\textnormal{терм.}}(x)
            =
            \| e_{\textnormal{схват}}(x) - e^{\textnormal{final}}\|^2
            +
            w \| \dot e_{\textnormal{схват}}(x) - \dot e^{\textnormal{final}}\|^2
        $$
        \begin{figure}
            \centering{
                \includegraphics[width=0.45\textwidth]{img/examples/reaching_speed_pendulum}
                \includegraphics[width=0.45\textwidth]{img/examples/reaching_speed_endpoint}
            }
            \caption{Траектории руки и схвата для задачи.}
        \end{figure}
    \end{frame}

    \begin{frame}{Задача: обход препятствия}
        $$
            \begin{aligned}
            & q_{\textnormal{терм.}}(x) = \| x - x^{\textnormal{final}}\|^2
            \\
            & q_{\textnormal{фаз.}}(x) = \left(\| e_{\textnormal{cхват}}(x) - e_{\textnormal{препят}}\|^2 - r_{\textnormal{препят}}\right)^2
            \end{aligned}
        $$
        \begin{figure}
            \centering{
                \includegraphics[width=0.45\textwidth]{img/examples/obstacle_pendulum_less}
                \includegraphics[width=0.45\textwidth]{img/examples/obstacle_endpoint_less}
            }
            \caption{Траектории руки и схвата для задачи.}
        \end{figure}
    \end{frame}

    \section{Литература}
    \begin{frame}{Доклады}
        \begin{itemize}
            \item Егоров К. Ю., Востриков И. В. Математическое моделирование движения руки и поведенческих движений // Доклад науч. конф. <<Ломоносовские чтения>> (Москва, 4-14 апреля 2023).
        \end{itemize}
    \end{frame}

    \begin{frame}{Список литературы}
        \begin{enumerate}
            \item[1] Колюбин~С.\,А. Динамика робототехнических систем~// Учебное пособие. --- СПб.: Университет ИТМО, 2017.~--- 117 с.
            \item[2] E. Todorov, M. Jordan. Optimal feedback control as a theory of motor coordination~// Nature Neuroscience, Vol.5, No.11, 1226-1235, 2002.
            \item[3] Y. Uno, M. Kawato, R. Suzuki. Formation and control of optimal trajectory in human multijoint arm movement~--- minimum torque-change model~// Biological Cybernetics 61, 89-101, 1989.
            \item[4] B.D.O. Anderson, J.B. Moore. Optimal Control: Linear Quadratic Methods~// Prentice Hall, Upper Saddle River, 1990.
            \item[5] D. H. Jacobson. Differential dynamic programming methods for determining optimal control of non-linear systems~// University of London, 1967.
        \end{enumerate}
    \end{frame}

    \begin{frame}{Список литературы}
        \begin{enumerate}
            \item[6] E. Guechi, S. Bouzoualegh, Y. Zennir, S. Blažič. MPC Control and LQ Optimal Control of A Two-Link Robot Arm: A Comparative Study~// Machines 6, no. 3: 37, 2018.
            \item[7] A. Babazadeh, N. Sadati. Optimal control of multiple-arm robotic systems using gradient method~// IEEE Conference on Robotics, Automation and Mechatronics, Singapore, pp. 312-317 vol.1, 2004.
        \end{enumerate}
    \end{frame}
\end{document}

\usepackage{amsmath}  % Математические 
\usepackage{amssymb}  % формулы
\usepackage{graphicx}

% Show only current section (without subsection in the frame header)
\setbeamertemplate{headline}{%
  \leavevmode%
  \begin{beamercolorbox}[wd=\paperwidth,ht=2.5ex,dp=1.5ex]{section in head/foot}%
    %\hfill
    \hspace{1em}\strut\insertsectionhead\hspace{.5em}\mbox{}%
  \end{beamercolorbox}%
  %\begin{beamercolorbox}[wd=.5\paperwidth,ht=2.5ex,dp=1.5ex]{subsection in head/foot}%
  %  \mbox{}\hspace{.5em}\strut\insertsubsectionhead\hfill%
  %\end{beamercolorbox}%
}

\title[Магистерская диссертация]
        {Математическое моделирование движений руки, держащей предмет}
\author[К. Ю. Егоров]
        {студент 2 курса магистратуры К. Ю. Егоров\\
        научный руководитель --- к.ф-м.н., доцент И. В. Востриков}
\institute{Кафедра системного анализа\\ ВМК МГУ}
\date{26 апреля 2023}

\begin{document}
    \begin{frame}
        \titlepage
    \end{frame}

    \begin{frame}{Содержание}
        \tableofcontents
    \end{frame}


    \section{Построение математической модели}

    \begin{frame}{Математическое моделирование}
        \centering{
            \hfill
            \includegraphics[width=0.3\textwidth]{img/model/arm.jpeg}
            \hfill
            \includegraphics[width=0.3\textwidth]{img/model/schema.png}
            \hfill
        }
        \begin{block}{Начальные данные}
            \begin{itemize}
                \item Рука --- 3-х сочленённый математический маятник.
                \item Известны длины $\ell_i$, массы $m_i$ и моменты инерции $I_i$.
                \item Фазовые переменные --- углы поворота сочленения $\theta_i$ относительно оси $Ox$. 
            \end{itemize}
        \end{block}
    \end{frame}

    \begin{frame}{Уравнение динамики}
        Метод Эйлера--Лагранжа
        $$
            \mathcal{L} = \Pi - K
            \;\Longrightarrow\;
            \tau_i
            =
              \frac{d}{dt}\left(\frac{\partial \mathcal{L}}{\partial \dot \theta_i}\right)
            - \frac{\partial \mathcal{L}}{\partial \theta_i},
        $$
        где $K$ и $\Pi$ --- общие кинетическая и потенциальная энергии системы.
        \vfill
        \begin{block}{Уравнение динамики}
            $$
                \tau = M(\theta)\ddot\theta + L(\theta, \dot\theta)
            $$
            \begin{itemize}
                \item $\tau_i$~--- момент силы, действующей на $i$-е сочленение
                \item $M(\theta)=M^{\textnormal{T}}(\theta)>0$~--- матрица инерции
                \item $L(\theta, \dot\theta)$~--- вектор центростремительных и корелисовых сил
            \end{itemize}
        \end{block}
    \end{frame}

    \begin{frame}{Математическое моделирование}
        \begin{figure}
            \includegraphics[width=0.6\textwidth]{img/model/pendulum.pdf}
            \caption{Траектория руки в свободном падении}
        \end{figure}
    \end{frame}

    \begin{frame}{Принцип оптимальности}
        Моторно-двигательная система как результат эволюции и обучения строит движение в соотвествии с принципом оптимальности.
        \begin{block}{Биологический принцип оптимальности}
            Выбираемые нервной системой схемы движения являются оптимальными для поставленной задачи.
        \end{block}
        \vfill
        В работе [2] показано, что оптимизации проводится с целью уменьшения затрат энергии.
        \begin{block}{Формализация энергетических затрат [3]}
            $$
                \mbox{Затраты}\, = \int\limits_{t_{start}}^{t_{final}}\|\dot\tau\|^2\,dt
            $$
        \end{block}
    \end{frame}


    \section{Постановка задачи оптимального управления}

    \begin{frame}{Постановка задачи}
        Введем $x = [\theta\;\dot\theta\;\tau]^{\textnormal{T}}$, тогда уравнение динамики примет вид
        $$
            \dot x = A(x) + Bu,\;
            \mbox{где}\,
            A(x) = \left[\begin{aligned}
                x_2 \\
                M^{-1}(x_1)(x_3 - L(x_1, x_2)) \\
                O
            \end{aligned}\right],
            B = \left[\begin{aligned}
                0 \\ 0 \\ I
            \end{aligned}\right].
        $$
        Задано начальное положение:
        $$
            x(t_{\textnormal{start}}) = x^{\textnormal{start}}
        $$
        Задача минимизации функционала:
        $$
            J
            =
            q_{\textnormal{терм.}}(x^{\textnormal{final}})
            +
            w_1\int_{t_{\textnormal{start}}}^{t_{\textnormal{final}}} q_{\textnormal{фаз.}}(x)\,dt
            +
            w_2\int_{t_{\textnormal{start}}}^{t_{\textnormal{final}}} \| u \|^2 \,dt
            \longrightarrow 
            \textnormal{min},
        $$
        где $q_{\textnormal{терм.}}$, $q_{\textnormal{фаз.}}$ выбираются исходя из конкретной постановки задачи.
    \end{frame}

    \begin{frame}{Дискретный вариант задачи}
        Введем сетку по переменной $t = \{t_k\}_{k = 1}^{N+1}$ с шагом $\Delta t$, тогда
        $$
            \left\{
            \begin{aligned}
            &x^{k+1} = f(x^k, u^k) \\
            &x^0 = x^{start},
            \end{aligned}
            \right.
        $$
        где $f(x^k, u^k) = \Delta t [A(x^k) + Bu^k] + x^k$.
        $$
            J
            =
            q^{N+1}(x^{N+1})
            +
            \sum_{k=1}^{N} q^{k}(x^k)
            +
            \sum_{k=1}^{N} r^{k}(u^k)
            \;\longrightarrow\;\textnormal{min},
        $$
        $q^{N+1}(x) = q_{\textnormal{терм.}}(x)$, $q^{k}(x) = w_1 \Delta t q_{\textnormal{фаз.}}(x)$, $r^k(u) = w_2 \Delta t \| u \|^2$.
    \end{frame}


    \section{Алгоритм синтеза управления}

    \begin{frame}{Алгоритм синтеза управления}
        \begin{block}{Идея метода}
            \begin{enumerate}
                \item Пусть на $k$ шаге алгоритма дано \textit{референсное} управление $\bar u$ и соответствующая ему траектория $\bar x$.
                \item Вдоль траектории линеаризуем систему, квадратично аппроксимируем функционал качества.
                \item Построим оптимальную поправку на управление {\centering $\delta u = \delta u(\bar x, \bar u)$ как линейно-квадратичный регулятор аппроксимированной задачи.}
                \item Если поправка выходит за допустимую область, то есть 
                $\xi_1
            \leqslant
            \frac{J(\bar u) - J(\bar u + \delta u^{*}(\eta))}{J_{\delta}(0) - J_{\delta}(\delta u^{*}(\eta))}
            \leqslant
            \xi_2$
                , то берем в качестве поправки $\eta\delta u$.
                \item Если мы не достигли требуемой точности, то есть $|J(\bar u) - J(\bar u + \delta u)| \geqslant \varepsilon$, повторяем действия с новым референсным управлением.
            \end{enumerate}
        \end{block}
    \end{frame}

    \begin{frame}{Аппроксимированная задача}
        Допустим мы имеем некоторое референсное управление~$\bar u$ и соответствующую ему референсную траекторию~$\bar x$.
        \begin{equation}\label{eq:ref-system}
            \left\{\begin{aligned}
                &\delta x^{k+1} = f_x^k \delta x + f_u^k \delta u \\
                &\delta x^0 = 0.
            \end{aligned}\right.
        \end{equation}
        \begin{multline}\label{eq:ref-cost}
            J = q^{N+1} + q_x^{N+1}\tilde x^{N+1} + \frac{1}{2}\langle \tilde x^{N+1}, q_{xx}^{N+1}\tilde x^{N+1} \rangle
            + \\ +
            \sum_{k=1}^{N}\left[ q^{k} + q_x^{k}\tilde x^{k} + \frac{1}{2}\langle \tilde x^{k}, q_{xx}^{k}\tilde x^{k} \rangle \right]
            + \\ +
            \sum_{k=1}^{N}\left[ r^{k} + r_u^{k}\tilde u^{k} + \frac{1}{2}\langle \tilde u^{k}, r_{uu}^{k}\tilde u^{k} \rangle \right],
        \end{multline}
        где $\tilde x^k = \bar x^k + \delta x^k$, $\tilde u^k = \bar u^k + \delta u^k$.
    \end{frame}

    \begin{frame}{Синтез управления}
        \begin{block}{Теорема [Об оптимальной поправке]}
            Оптимальная поправка $\delta u$ для задачи~\eqref{eq:ref-system}-\eqref{eq:ref-cost} вычисляется как
            \begin{equation*}
                \delta u^k = - (r^k_{uu} + (f^k_u)^{\textnormal{T}} S_{k+1} f^k_u)^{-1} ((f^k_u)^{\textnormal{T}} S_{k+1} f^k_u \delta x + v^{k+1} + r^k_u),
            \end{equation*}
            где $S_k$ и $v^{k}$ высчитываются в обратном времени как
            \begin{equation*}
                    S_k = q^k_{xx} + (f^k_x)^{\textnormal{T}} S_{k+1} (I + f^k_u (r^k_{uu})^{-1} (f^k_u)^{\textnormal{T}} S_{k+1})^{-1} f_x^k,
            \end{equation*}
            \begin{multline*}
                v^k = (f^k_x)^{\textnormal{T}} S_{k+1} ( I + f^k_u (r^k_{uu})^{-1} (f^k_u)^{\textnormal{T}} S_{k+1})^{-1}
                \cdot \\ \cdot
                (-f^k_u (r^k_{uu})^{-1} (f^k_u)^{\textnormal{T}} v^{k+1} + f^k_u (r^k_{uu})^{-1} r^k_u) + (f^k_x)^{\textnormal{T}} v^{k+1} + q^k_x
            \end{multline*}
            с граничными условиями
            \begin{equation*}
                    S_{N+1} = q^{N+1}_{xx},
                    \quad
                    v^{N+1} = q^{N+1}_{x}.
            \end{equation*}
        \end{block}
    \end{frame}
    
    \begin{frame}{Пример: Целевое состояние}
        $$
            q_{\textnormal{терм.}}(x) = \| x - x^{\textnormal{final}}\|^2
        $$
        \begin{figure}
            \centering{
                \includegraphics[width=0.45\textwidth]{img/iLQR/pendulum.pdf}
                \includegraphics[width=0.45\textwidth]{img/iLQR/endpoint.pdf}
            }
            \caption{Траектории руки и схвата для задачи при нулевом начальном управлении $\bar u \equiv 0$.}
        \end{figure}
    \end{frame}


    \section{Выбор начальной траектории алгоритма}

    \begin{frame}{Выбор начальной траектории}
        \begin{block}{Мотивация}
            Предложенный метод сходится медленно. Построим начальное референсное управление $\bar u$ так, чтобы:
            \begin{itemize}
                \item Оно удовлетворяло терминальному условию;
                \item Оно \textit{быстро} строилось;
                \item Оно было допустимо для рассматриваемой задачи.
            \end{itemize}
        \end{block}
    \end{frame}

    \begin{frame}{Референсная траектория}
        Приведем систему к линейной, заменой управления
        $$
            v = M^{-1}(x_1)(\tau - L(x_1, x_2))
        $$
        Тогда система примет вид:
        \begin{equation}\label{eq:linear}
            x^{k+1} =  A x^{k} + B v^{k}
        \end{equation}

        Выберем $x^{\textnormal{final}} \in \textnormal{Argmin}_{x} q^{N+1}(x)$ и решим задачу минимизации

        \begin{equation}\label{eq:linear-cost}
            J = \| x-x^{\textnormal{final}} \|^2 + w \sum_{k=1}^{N} \| v^k \|^2 \longrightarrow \textnormal{min}
        \end{equation}

        Получим референсное управление $u$ из соотношения
        $$
            \tau^k = M(x_1^k)v^k_* + L(x_1^k,x_2^k) \;\Longrightarrow\; u^{k} = \frac{\tau^{k} - \tau^{k-1}}{\Delta t}
        $$
    \end{frame}

    \begin{frame}{Референсная траектория}
        Метод динамического программирования даёт решение задачи.
        \vfill
        \begin{block}{Утверждение}
            Оптимальное управление $v_*$ задачи \eqref{eq:linear}-\eqref{eq:linear-cost} высчитывается как
            $$
                v^k_* = -[wI + B^{\textnormal{T}}P^kB]^{-1} P^k B A x,
            $$
            где матрица $P^k$ может быть посчитана в обратном времени как решение уравнения Риккати
            $$
                \begin{aligned}
                    &P^{k-1} = A^{\textnormal{T}} P^{k} A - A^{\textnormal{T}} P^k B [wI + B^{\textnormal{T}} P^k B]^{-1} B^{\textnormal{T}} P^{k} A
                    \\
                    &P^{N+1} = I
                \end{aligned}
            $$
        \end{block}
    \end{frame}

    \begin{frame}{Пример: целевое состояние}
        $$
            q_{\textnormal{терм.}}(x) = \| x - x^{\textnormal{final}}\|^2
        $$
        \begin{figure}
            \centering{
                \includegraphics[width=0.45\textwidth]{img/initial/pendulum.pdf}
                \includegraphics[width=0.45\textwidth]{img/initial/endpoint.pdf}
            }
            \caption{Траектории руки и схвата для задачи при новом начальном управлении.}
        \end{figure}
    \end{frame}

    \begin{frame}{Пример: целевое состояние}
        $$
            q_{\textnormal{терм.}}(x) = \| x - x^{\textnormal{final}}\|^2
        $$
        \begin{figure}
            \centering{
                \includegraphics[width=0.6\textwidth]{img/initial/compare}
            }
            \caption{Сравнение скорости сходимости алгоритма в зависимости от начальной траектории.}
        \end{figure}
    \end{frame}


    \section{Применение в классических задачах}

    \begin{frame}{Задача: целевое положение схвата}
        $$
            q_{\textnormal{терм.}}(x) = \| e_{\textnormal{схват}}(x) - e^{\textnormal{final}}\|^2
        $$
        \begin{figure}
            \centering{
                \includegraphics[width=0.45\textwidth]{img/examples/reaching_pendulum}
                \includegraphics[width=0.45\textwidth]{img/examples/reaching_endpoint}
            }
            \caption{Траектории руки и схвата для задачи.}
        \end{figure}
    \end{frame}

    \begin{frame}{Задача: целевое положение и скорость схвата}
        $$
            q_{\textnormal{терм.}}(x)
            =
            \| e_{\textnormal{схват}}(x) - e^{\textnormal{final}}\|^2
            +
            w \| \dot e_{\textnormal{схват}}(x) - \dot e^{\textnormal{final}}\|^2
        $$
        \begin{figure}
            \centering{
                \includegraphics[width=0.45\textwidth]{img/examples/reaching_speed_pendulum}
                \includegraphics[width=0.45\textwidth]{img/examples/reaching_speed_endpoint}
            }
            \caption{Траектории руки и схвата для задачи.}
        \end{figure}
    \end{frame}

    \begin{frame}{Задача: обход препятствия}
        $$
            \begin{aligned}
            & q_{\textnormal{терм.}}(x) = \| x - x^{\textnormal{final}}\|^2
            \\
            & q_{\textnormal{фаз.}}(x) = \left(\| e_{\textnormal{cхват}}(x) - e_{\textnormal{препят}}\|^2 - r_{\textnormal{препят}}\right)^2
            \end{aligned}
        $$
        \begin{figure}
            \centering{
                \includegraphics[width=0.45\textwidth]{img/examples/obstacle_pendulum_less}
                \includegraphics[width=0.45\textwidth]{img/examples/obstacle_endpoint_less}
            }
            \caption{Траектории руки и схвата для задачи.}
        \end{figure}
    \end{frame}

    \section{Литература}
    \begin{frame}{Доклады}
        \begin{itemize}
            \item Егоров К. Ю., Востриков И. В. Математическое моделирование движения руки и поведенческих движений // Доклад науч. конф. <<Ломоносовские чтения>> (Москва, 4-14 апреля 2023).
        \end{itemize}
    \end{frame}

    \begin{frame}{Список литературы}
        \begin{enumerate}
            \item[1] Колюбин~С.\,А. Динамика робототехнических систем~// Учебное пособие. --- СПб.: Университет ИТМО, 2017.~--- 117 с.
            \item[2] E. Todorov, M. Jordan. Optimal feedback control as a theory of motor coordination~// Nature Neuroscience, Vol.5, No.11, 1226-1235, 2002.
            \item[3] Y. Uno, M. Kawato, R. Suzuki. Formation and control of optimal trajectory in human multijoint arm movement~--- minimum torque-change model~// Biological Cybernetics 61, 89-101, 1989.
            \item[4] B.D.O. Anderson, J.B. Moore. Optimal Control: Linear Quadratic Methods~// Prentice Hall, Upper Saddle River, 1990.
            \item[5] D. H. Jacobson. Differential dynamic programming methods for determining optimal control of non-linear systems~// University of London, 1967.
        \end{enumerate}
    \end{frame}

    \begin{frame}{Список литературы}
        \begin{enumerate}
            \item[6] E. Guechi, S. Bouzoualegh, Y. Zennir, S. Blažič. MPC Control and LQ Optimal Control of A Two-Link Robot Arm: A Comparative Study~// Machines 6, no. 3: 37, 2018.
            \item[7] A. Babazadeh, N. Sadati. Optimal control of multiple-arm robotic systems using gradient method~// IEEE Conference on Robotics, Automation and Mechatronics, Singapore, pp. 312-317 vol.1, 2004.
        \end{enumerate}
    \end{frame}
\end{document}

\usepackage{amsmath}  % Математические 
\usepackage{amssymb}  % формулы
\usepackage{graphicx}

% Show only current section (without subsection in the frame header)
\setbeamertemplate{headline}{%
  \leavevmode%
  \begin{beamercolorbox}[wd=\paperwidth,ht=2.5ex,dp=1.5ex]{section in head/foot}%
    %\hfill
    \hspace{1em}\strut\insertsectionhead\hspace{.5em}\mbox{}%
  \end{beamercolorbox}%
  %\begin{beamercolorbox}[wd=.5\paperwidth,ht=2.5ex,dp=1.5ex]{subsection in head/foot}%
  %  \mbox{}\hspace{.5em}\strut\insertsubsectionhead\hfill%
  %\end{beamercolorbox}%
}

\title[Магистерская диссертация]
        {Математическое моделирование движений руки, держащей предмет}
\author[К. Ю. Егоров]
        {студент 2 курса магистратуры К. Ю. Егоров\\
        научный руководитель --- к.ф-м.н., доцент И. В. Востриков}
\institute{Кафедра системного анализа\\ ВМК МГУ}
\date{26 апреля 2023}

\begin{document}
    \begin{frame}
        \titlepage
    \end{frame}

    \begin{frame}{Содержание}
        \tableofcontents
    \end{frame}


    \section{Построение математической модели}

    \begin{frame}{Математическое моделирование}
        \centering{
            \hfill
            \includegraphics[width=0.3\textwidth]{img/model/arm.jpeg}
            \hfill
            \includegraphics[width=0.3\textwidth]{img/model/schema.png}
            \hfill
        }
        \begin{block}{Начальные данные}
            \begin{itemize}
                \item Рука --- 3-х сочленённый математический маятник.
                \item Известны длины $\ell_i$, массы $m_i$ и моменты инерции $I_i$.
                \item Фазовые переменные --- углы поворота сочленения $\theta_i$ относительно оси $Ox$. 
            \end{itemize}
        \end{block}
    \end{frame}

    \begin{frame}{Уравнение динамики}
        Метод Эйлера--Лагранжа
        $$
            \mathcal{L} = \Pi - K
            \;\Longrightarrow\;
            \tau_i
            =
              \frac{d}{dt}\left(\frac{\partial \mathcal{L}}{\partial \dot \theta_i}\right)
            - \frac{\partial \mathcal{L}}{\partial \theta_i},
        $$
        где $K$ и $\Pi$ --- общие кинетическая и потенциальная энергии системы.
        \vfill
        \begin{block}{Уравнение динамики}
            $$
                \tau = M(\theta)\ddot\theta + L(\theta, \dot\theta)
            $$
            \begin{itemize}
                \item $\tau_i$~--- момент силы, действующей на $i$-е сочленение
                \item $M(\theta)=M^{\textnormal{T}}(\theta)>0$~--- матрица инерции
                \item $L(\theta, \dot\theta)$~--- вектор центростремительных и корелисовых сил
            \end{itemize}
        \end{block}
    \end{frame}

    \begin{frame}{Математическое моделирование}
        \begin{figure}
            \includegraphics[width=0.6\textwidth]{img/model/pendulum.pdf}
            \caption{Траектория руки в свободном падении}
        \end{figure}
    \end{frame}

    \begin{frame}{Принцип оптимальности}
        Моторно-двигательная система как результат эволюции и обучения строит движение в соотвествии с принципом оптимальности.
        \begin{block}{Биологический принцип оптимальности}
            Выбираемые нервной системой схемы движения являются оптимальными для поставленной задачи.
        \end{block}
        \vfill
        В работе [2] показано, что оптимизации проводится с целью уменьшения затрат энергии.
        \begin{block}{Формализация энергетических затрат [3]}
            $$
                \mbox{Затраты}\, = \int\limits_{t_{start}}^{t_{final}}\|\dot\tau\|^2\,dt
            $$
        \end{block}
    \end{frame}


    \section{Постановка задачи оптимального управления}

    \begin{frame}{Постановка задачи}
        Введем $x = [\theta\;\dot\theta\;\tau]^{\textnormal{T}}$, тогда уравнение динамики примет вид
        $$
            \dot x = A(x) + Bu,\;
            \mbox{где}\,
            A(x) = \left[\begin{aligned}
                x_2 \\
                M^{-1}(x_1)(x_3 - L(x_1, x_2)) \\
                O
            \end{aligned}\right],
            B = \left[\begin{aligned}
                0 \\ 0 \\ I
            \end{aligned}\right].
        $$
        Задано начальное положение:
        $$
            x(t_{\textnormal{start}}) = x^{\textnormal{start}}
        $$
        Задача минимизации функционала:
        $$
            J
            =
            q_{\textnormal{терм.}}(x^{\textnormal{final}})
            +
            w_1\int_{t_{\textnormal{start}}}^{t_{\textnormal{final}}} q_{\textnormal{фаз.}}(x)\,dt
            +
            w_2\int_{t_{\textnormal{start}}}^{t_{\textnormal{final}}} \| u \|^2 \,dt
            \longrightarrow 
            \textnormal{min},
        $$
        где $q_{\textnormal{терм.}}$, $q_{\textnormal{фаз.}}$ выбираются исходя из конкретной постановки задачи.
    \end{frame}

    \begin{frame}{Дискретный вариант задачи}
        Введем сетку по переменной $t = \{t_k\}_{k = 1}^{N+1}$ с шагом $\Delta t$, тогда
        $$
            \left\{
            \begin{aligned}
            &x^{k+1} = f(x^k, u^k) \\
            &x^0 = x^{start},
            \end{aligned}
            \right.
        $$
        где $f(x^k, u^k) = \Delta t [A(x^k) + Bu^k] + x^k$.
        $$
            J
            =
            q^{N+1}(x^{N+1})
            +
            \sum_{k=1}^{N} q^{k}(x^k)
            +
            \sum_{k=1}^{N} r^{k}(u^k)
            \;\longrightarrow\;\textnormal{min},
        $$
        $q^{N+1}(x) = q_{\textnormal{терм.}}(x)$, $q^{k}(x) = w_1 \Delta t q_{\textnormal{фаз.}}(x)$, $r^k(u) = w_2 \Delta t \| u \|^2$.
    \end{frame}


    \section{Алгоритм синтеза управления}

    \begin{frame}{Алгоритм синтеза управления}
        \begin{block}{Идея метода}
            \begin{enumerate}
                \item Пусть на $k$ шаге алгоритма дано \textit{референсное} управление $\bar u$ и соответствующая ему траектория $\bar x$.
                \item Вдоль траектории линеаризуем систему, квадратично аппроксимируем функционал качества.
                \item Построим оптимальную поправку на управление {\centering $\delta u = \delta u(\bar x, \bar u)$ как линейно-квадратичный регулятор аппроксимированной задачи.}
                \item Если поправка выходит за допустимую область, то есть 
                $\xi_1
            \leqslant
            \frac{J(\bar u) - J(\bar u + \delta u^{*}(\eta))}{J_{\delta}(0) - J_{\delta}(\delta u^{*}(\eta))}
            \leqslant
            \xi_2$
                , то берем в качестве поправки $\eta\delta u$.
                \item Если мы не достигли требуемой точности, то есть $|J(\bar u) - J(\bar u + \delta u)| \geqslant \varepsilon$, повторяем действия с новым референсным управлением.
            \end{enumerate}
        \end{block}
    \end{frame}

    \begin{frame}{Аппроксимированная задача}
        Допустим мы имеем некоторое референсное управление~$\bar u$ и соответствующую ему референсную траекторию~$\bar x$.
        \begin{equation}\label{eq:ref-system}
            \left\{\begin{aligned}
                &\delta x^{k+1} = f_x^k \delta x + f_u^k \delta u \\
                &\delta x^0 = 0.
            \end{aligned}\right.
        \end{equation}
        \begin{multline}\label{eq:ref-cost}
            J = q^{N+1} + q_x^{N+1}\tilde x^{N+1} + \frac{1}{2}\langle \tilde x^{N+1}, q_{xx}^{N+1}\tilde x^{N+1} \rangle
            + \\ +
            \sum_{k=1}^{N}\left[ q^{k} + q_x^{k}\tilde x^{k} + \frac{1}{2}\langle \tilde x^{k}, q_{xx}^{k}\tilde x^{k} \rangle \right]
            + \\ +
            \sum_{k=1}^{N}\left[ r^{k} + r_u^{k}\tilde u^{k} + \frac{1}{2}\langle \tilde u^{k}, r_{uu}^{k}\tilde u^{k} \rangle \right],
        \end{multline}
        где $\tilde x^k = \bar x^k + \delta x^k$, $\tilde u^k = \bar u^k + \delta u^k$.
    \end{frame}

    \begin{frame}{Синтез управления}
        \begin{block}{Теорема [Об оптимальной поправке]}
            Оптимальная поправка $\delta u$ для задачи~\eqref{eq:ref-system}-\eqref{eq:ref-cost} вычисляется как
            \begin{equation*}
                \delta u^k = - (r^k_{uu} + (f^k_u)^{\textnormal{T}} S_{k+1} f^k_u)^{-1} ((f^k_u)^{\textnormal{T}} S_{k+1} f^k_u \delta x + v^{k+1} + r^k_u),
            \end{equation*}
            где $S_k$ и $v^{k}$ высчитываются в обратном времени как
            \begin{equation*}
                    S_k = q^k_{xx} + (f^k_x)^{\textnormal{T}} S_{k+1} (I + f^k_u (r^k_{uu})^{-1} (f^k_u)^{\textnormal{T}} S_{k+1})^{-1} f_x^k,
            \end{equation*}
            \begin{multline*}
                v^k = (f^k_x)^{\textnormal{T}} S_{k+1} ( I + f^k_u (r^k_{uu})^{-1} (f^k_u)^{\textnormal{T}} S_{k+1})^{-1}
                \cdot \\ \cdot
                (-f^k_u (r^k_{uu})^{-1} (f^k_u)^{\textnormal{T}} v^{k+1} + f^k_u (r^k_{uu})^{-1} r^k_u) + (f^k_x)^{\textnormal{T}} v^{k+1} + q^k_x
            \end{multline*}
            с граничными условиями
            \begin{equation*}
                    S_{N+1} = q^{N+1}_{xx},
                    \quad
                    v^{N+1} = q^{N+1}_{x}.
            \end{equation*}
        \end{block}
    \end{frame}
    
    \begin{frame}{Пример: Целевое состояние}
        $$
            q_{\textnormal{терм.}}(x) = \| x - x^{\textnormal{final}}\|^2
        $$
        \begin{figure}
            \centering{
                \includegraphics[width=0.45\textwidth]{img/iLQR/pendulum.pdf}
                \includegraphics[width=0.45\textwidth]{img/iLQR/endpoint.pdf}
            }
            \caption{Траектории руки и схвата для задачи при нулевом начальном управлении $\bar u \equiv 0$.}
        \end{figure}
    \end{frame}


    \section{Выбор начальной траектории алгоритма}

    \begin{frame}{Выбор начальной траектории}
        \begin{block}{Мотивация}
            Предложенный метод сходится медленно. Построим начальное референсное управление $\bar u$ так, чтобы:
            \begin{itemize}
                \item Оно удовлетворяло терминальному условию;
                \item Оно \textit{быстро} строилось;
                \item Оно было допустимо для рассматриваемой задачи.
            \end{itemize}
        \end{block}
    \end{frame}

    \begin{frame}{Референсная траектория}
        Приведем систему к линейной, заменой управления
        $$
            v = M^{-1}(x_1)(\tau - L(x_1, x_2))
        $$
        Тогда система примет вид:
        \begin{equation}\label{eq:linear}
            x^{k+1} =  A x^{k} + B v^{k}
        \end{equation}

        Выберем $x^{\textnormal{final}} \in \textnormal{Argmin}_{x} q^{N+1}(x)$ и решим задачу минимизации

        \begin{equation}\label{eq:linear-cost}
            J = \| x-x^{\textnormal{final}} \|^2 + w \sum_{k=1}^{N} \| v^k \|^2 \longrightarrow \textnormal{min}
        \end{equation}

        Получим референсное управление $u$ из соотношения
        $$
            \tau^k = M(x_1^k)v^k_* + L(x_1^k,x_2^k) \;\Longrightarrow\; u^{k} = \frac{\tau^{k} - \tau^{k-1}}{\Delta t}
        $$
    \end{frame}

    \begin{frame}{Референсная траектория}
        Метод динамического программирования даёт решение задачи.
        \vfill
        \begin{block}{Утверждение}
            Оптимальное управление $v_*$ задачи \eqref{eq:linear}-\eqref{eq:linear-cost} высчитывается как
            $$
                v^k_* = -[wI + B^{\textnormal{T}}P^kB]^{-1} P^k B A x,
            $$
            где матрица $P^k$ может быть посчитана в обратном времени как решение уравнения Риккати
            $$
                \begin{aligned}
                    &P^{k-1} = A^{\textnormal{T}} P^{k} A - A^{\textnormal{T}} P^k B [wI + B^{\textnormal{T}} P^k B]^{-1} B^{\textnormal{T}} P^{k} A
                    \\
                    &P^{N+1} = I
                \end{aligned}
            $$
        \end{block}
    \end{frame}

    \begin{frame}{Пример: целевое состояние}
        $$
            q_{\textnormal{терм.}}(x) = \| x - x^{\textnormal{final}}\|^2
        $$
        \begin{figure}
            \centering{
                \includegraphics[width=0.45\textwidth]{img/initial/pendulum.pdf}
                \includegraphics[width=0.45\textwidth]{img/initial/endpoint.pdf}
            }
            \caption{Траектории руки и схвата для задачи при новом начальном управлении.}
        \end{figure}
    \end{frame}

    \begin{frame}{Пример: целевое состояние}
        $$
            q_{\textnormal{терм.}}(x) = \| x - x^{\textnormal{final}}\|^2
        $$
        \begin{figure}
            \centering{
                \includegraphics[width=0.6\textwidth]{img/initial/compare}
            }
            \caption{Сравнение скорости сходимости алгоритма в зависимости от начальной траектории.}
        \end{figure}
    \end{frame}


    \section{Применение в классических задачах}

    \begin{frame}{Задача: целевое положение схвата}
        $$
            q_{\textnormal{терм.}}(x) = \| e_{\textnormal{схват}}(x) - e^{\textnormal{final}}\|^2
        $$
        \begin{figure}
            \centering{
                \includegraphics[width=0.45\textwidth]{img/examples/reaching_pendulum}
                \includegraphics[width=0.45\textwidth]{img/examples/reaching_endpoint}
            }
            \caption{Траектории руки и схвата для задачи.}
        \end{figure}
    \end{frame}

    \begin{frame}{Задача: целевое положение и скорость схвата}
        $$
            q_{\textnormal{терм.}}(x)
            =
            \| e_{\textnormal{схват}}(x) - e^{\textnormal{final}}\|^2
            +
            w \| \dot e_{\textnormal{схват}}(x) - \dot e^{\textnormal{final}}\|^2
        $$
        \begin{figure}
            \centering{
                \includegraphics[width=0.45\textwidth]{img/examples/reaching_speed_pendulum}
                \includegraphics[width=0.45\textwidth]{img/examples/reaching_speed_endpoint}
            }
            \caption{Траектории руки и схвата для задачи.}
        \end{figure}
    \end{frame}

    \begin{frame}{Задача: обход препятствия}
        $$
            \begin{aligned}
            & q_{\textnormal{терм.}}(x) = \| x - x^{\textnormal{final}}\|^2
            \\
            & q_{\textnormal{фаз.}}(x) = \left(\| e_{\textnormal{cхват}}(x) - e_{\textnormal{препят}}\|^2 - r_{\textnormal{препят}}\right)^2
            \end{aligned}
        $$
        \begin{figure}
            \centering{
                \includegraphics[width=0.45\textwidth]{img/examples/obstacle_pendulum_less}
                \includegraphics[width=0.45\textwidth]{img/examples/obstacle_endpoint_less}
            }
            \caption{Траектории руки и схвата для задачи.}
        \end{figure}
    \end{frame}

    \section{Литература}
    \begin{frame}{Доклады}
        \begin{itemize}
            \item Егоров К. Ю., Востриков И. В. Математическое моделирование движения руки и поведенческих движений // Доклад науч. конф. <<Ломоносовские чтения>> (Москва, 4-14 апреля 2023).
        \end{itemize}
    \end{frame}

    \begin{frame}{Список литературы}
        \begin{enumerate}
            \item[1] Колюбин~С.\,А. Динамика робототехнических систем~// Учебное пособие. --- СПб.: Университет ИТМО, 2017.~--- 117 с.
            \item[2] E. Todorov, M. Jordan. Optimal feedback control as a theory of motor coordination~// Nature Neuroscience, Vol.5, No.11, 1226-1235, 2002.
            \item[3] Y. Uno, M. Kawato, R. Suzuki. Formation and control of optimal trajectory in human multijoint arm movement~--- minimum torque-change model~// Biological Cybernetics 61, 89-101, 1989.
            \item[4] B.D.O. Anderson, J.B. Moore. Optimal Control: Linear Quadratic Methods~// Prentice Hall, Upper Saddle River, 1990.
            \item[5] D. H. Jacobson. Differential dynamic programming methods for determining optimal control of non-linear systems~// University of London, 1967.
        \end{enumerate}
    \end{frame}

    \begin{frame}{Список литературы}
        \begin{enumerate}
            \item[6] E. Guechi, S. Bouzoualegh, Y. Zennir, S. Blažič. MPC Control and LQ Optimal Control of A Two-Link Robot Arm: A Comparative Study~// Machines 6, no. 3: 37, 2018.
            \item[7] A. Babazadeh, N. Sadati. Optimal control of multiple-arm robotic systems using gradient method~// IEEE Conference on Robotics, Automation and Mechatronics, Singapore, pp. 312-317 vol.1, 2004.
        \end{enumerate}
    \end{frame}
\end{document}

\usepackage{amsmath}  % Математические 
\usepackage{amssymb}  % формулы
\usepackage{graphicx}

% Show only current section (without subsection in the frame header)
\setbeamertemplate{headline}{%
  \leavevmode%
  \begin{beamercolorbox}[wd=\paperwidth,ht=2.5ex,dp=1.5ex]{section in head/foot}%
    %\hfill
    \hspace{1em}\strut\insertsectionhead\hspace{.5em}\mbox{}%
  \end{beamercolorbox}%
  %\begin{beamercolorbox}[wd=.5\paperwidth,ht=2.5ex,dp=1.5ex]{subsection in head/foot}%
  %  \mbox{}\hspace{.5em}\strut\insertsubsectionhead\hfill%
  %\end{beamercolorbox}%
}

\title[Магистерская диссертация]
        {Математическое моделирование движений руки, держащей предмет}
\author[К. Ю. Егоров]
        {студент 2 курса магистратуры К. Ю. Егоров\\
        научный руководитель --- к.ф-м.н., доцент И. В. Востриков}
\institute{Кафедра системного анализа\\ ВМК МГУ}
\date{26 апреля 2023}

\begin{document}
    \begin{frame}
        \titlepage
    \end{frame}

    \begin{frame}{Содержание}
        \tableofcontents
    \end{frame}


    \section{Построение математической модели}

    \begin{frame}{Математическое моделирование}
        \centering{
            \hfill
            \includegraphics[width=0.3\textwidth]{img/model/arm.jpeg}
            \hfill
            \includegraphics[width=0.3\textwidth]{img/model/schema.png}
            \hfill
        }
        \begin{block}{Начальные данные}
            \begin{itemize}
                \item Рука --- 3-х сочленённый математический маятник.
                \item Известны длины $\ell_i$, массы $m_i$ и моменты инерции $I_i$.
                \item Фазовые переменные --- углы поворота сочленения $\theta_i$ относительно оси $Ox$. 
            \end{itemize}
        \end{block}
    \end{frame}

    \begin{frame}{Уравнение динамики}
        Метод Эйлера--Лагранжа
        $$
            \mathcal{L} = \Pi - K
            \;\Longrightarrow\;
            \tau_i
            =
              \frac{d}{dt}\left(\frac{\partial \mathcal{L}}{\partial \dot \theta_i}\right)
            - \frac{\partial \mathcal{L}}{\partial \theta_i},
        $$
        где $K$ и $\Pi$ --- общие кинетическая и потенциальная энергии системы.
        \vfill
        \begin{block}{Уравнение динамики}
            $$
                \tau = M(\theta)\ddot\theta + L(\theta, \dot\theta)
            $$
            \begin{itemize}
                \item $\tau_i$~--- момент силы, действующей на $i$-е сочленение
                \item $M(\theta)=M^{\textnormal{T}}(\theta)>0$~--- матрица инерции
                \item $L(\theta, \dot\theta)$~--- вектор центростремительных и корелисовых сил
            \end{itemize}
        \end{block}
    \end{frame}

    \begin{frame}{Математическое моделирование}
        \begin{figure}
            \includegraphics[width=0.6\textwidth]{img/model/pendulum.pdf}
            \caption{Траектория руки в свободном падении}
        \end{figure}
    \end{frame}

    \begin{frame}{Принцип оптимальности}
        Моторно-двигательная система как результат эволюции и обучения строит движение в соотвествии с принципом оптимальности.
        \begin{block}{Биологический принцип оптимальности}
            Выбираемые нервной системой схемы движения являются оптимальными для поставленной задачи.
        \end{block}
        \vfill
        В работе [2] показано, что оптимизации проводится с целью уменьшения затрат энергии.
        \begin{block}{Формализация энергетических затрат [3]}
            $$
                \mbox{Затраты}\, = \int\limits_{t_{start}}^{t_{final}}\|\dot\tau\|^2\,dt
            $$
        \end{block}
    \end{frame}


    \section{Постановка задачи оптимального управления}

    \begin{frame}{Постановка задачи}
        Введем $x = [\theta\;\dot\theta\;\tau]^{\textnormal{T}}$, тогда уравнение динамики примет вид
        $$
            \dot x = A(x) + Bu,\;
            \mbox{где}\,
            A(x) = \left[\begin{aligned}
                x_2 \\
                M^{-1}(x_1)(x_3 - L(x_1, x_2)) \\
                O
            \end{aligned}\right],
            B = \left[\begin{aligned}
                0 \\ 0 \\ I
            \end{aligned}\right].
        $$
        Задано начальное положение:
        $$
            x(t_{\textnormal{start}}) = x^{\textnormal{start}}
        $$
        Задача минимизации функционала:
        $$
            J
            =
            q_{\textnormal{терм.}}(x^{\textnormal{final}})
            +
            w_1\int_{t_{\textnormal{start}}}^{t_{\textnormal{final}}} q_{\textnormal{фаз.}}(x)\,dt
            +
            w_2\int_{t_{\textnormal{start}}}^{t_{\textnormal{final}}} \| u \|^2 \,dt
            \longrightarrow 
            \textnormal{min},
        $$
        где $q_{\textnormal{терм.}}$, $q_{\textnormal{фаз.}}$ выбираются исходя из конкретной постановки задачи.
    \end{frame}

    \begin{frame}{Дискретный вариант задачи}
        Введем сетку по переменной $t = \{t_k\}_{k = 1}^{N+1}$ с шагом $\Delta t$, тогда
        $$
            \left\{
            \begin{aligned}
            &x^{k+1} = f(x^k, u^k) \\
            &x^0 = x^{start},
            \end{aligned}
            \right.
        $$
        где $f(x^k, u^k) = \Delta t [A(x^k) + Bu^k] + x^k$.
        $$
            J
            =
            q^{N+1}(x^{N+1})
            +
            \sum_{k=1}^{N} q^{k}(x^k)
            +
            \sum_{k=1}^{N} r^{k}(u^k)
            \;\longrightarrow\;\textnormal{min},
        $$
        $q^{N+1}(x) = q_{\textnormal{терм.}}(x)$, $q^{k}(x) = w_1 \Delta t q_{\textnormal{фаз.}}(x)$, $r^k(u) = w_2 \Delta t \| u \|^2$.
    \end{frame}


    \section{Алгоритм синтеза управления}

    \begin{frame}{Алгоритм синтеза управления}
        \begin{block}{Идея метода}
            \begin{enumerate}
                \item Пусть на $k$ шаге алгоритма дано \textit{референсное} управление $\bar u$ и соответствующая ему траектория $\bar x$.
                \item Вдоль траектории линеаризуем систему, квадратично аппроксимируем функционал качества.
                \item Построим оптимальную поправку на управление {\centering $\delta u = \delta u(\bar x, \bar u)$ как линейно-квадратичный регулятор аппроксимированной задачи.}
                \item Если поправка выходит за допустимую область, то есть 
                $\xi_1
            \leqslant
            \frac{J(\bar u) - J(\bar u + \delta u^{*}(\eta))}{J_{\delta}(0) - J_{\delta}(\delta u^{*}(\eta))}
            \leqslant
            \xi_2$
                , то берем в качестве поправки $\eta\delta u$.
                \item Если мы не достигли требуемой точности, то есть $|J(\bar u) - J(\bar u + \delta u)| \geqslant \varepsilon$, повторяем действия с новым референсным управлением.
            \end{enumerate}
        \end{block}
    \end{frame}

    \begin{frame}{Аппроксимированная задача}
        Допустим мы имеем некоторое референсное управление~$\bar u$ и соответствующую ему референсную траекторию~$\bar x$.
        \begin{equation}\label{eq:ref-system}
            \left\{\begin{aligned}
                &\delta x^{k+1} = f_x^k \delta x + f_u^k \delta u \\
                &\delta x^0 = 0.
            \end{aligned}\right.
        \end{equation}
        \begin{multline}\label{eq:ref-cost}
            J = q^{N+1} + q_x^{N+1}\tilde x^{N+1} + \frac{1}{2}\langle \tilde x^{N+1}, q_{xx}^{N+1}\tilde x^{N+1} \rangle
            + \\ +
            \sum_{k=1}^{N}\left[ q^{k} + q_x^{k}\tilde x^{k} + \frac{1}{2}\langle \tilde x^{k}, q_{xx}^{k}\tilde x^{k} \rangle \right]
            + \\ +
            \sum_{k=1}^{N}\left[ r^{k} + r_u^{k}\tilde u^{k} + \frac{1}{2}\langle \tilde u^{k}, r_{uu}^{k}\tilde u^{k} \rangle \right],
        \end{multline}
        где $\tilde x^k = \bar x^k + \delta x^k$, $\tilde u^k = \bar u^k + \delta u^k$.
    \end{frame}

    \begin{frame}{Синтез управления}
        \begin{block}{Теорема [Об оптимальной поправке]}
            Оптимальная поправка $\delta u$ для задачи~\eqref{eq:ref-system}-\eqref{eq:ref-cost} вычисляется как
            \begin{equation*}
                \delta u^k = - (r^k_{uu} + (f^k_u)^{\textnormal{T}} S_{k+1} f^k_u)^{-1} ((f^k_u)^{\textnormal{T}} S_{k+1} f^k_u \delta x + v^{k+1} + r^k_u),
            \end{equation*}
            где $S_k$ и $v^{k}$ высчитываются в обратном времени как
            \begin{equation*}
                    S_k = q^k_{xx} + (f^k_x)^{\textnormal{T}} S_{k+1} (I + f^k_u (r^k_{uu})^{-1} (f^k_u)^{\textnormal{T}} S_{k+1})^{-1} f_x^k,
            \end{equation*}
            \begin{multline*}
                v^k = (f^k_x)^{\textnormal{T}} S_{k+1} ( I + f^k_u (r^k_{uu})^{-1} (f^k_u)^{\textnormal{T}} S_{k+1})^{-1}
                \cdot \\ \cdot
                (-f^k_u (r^k_{uu})^{-1} (f^k_u)^{\textnormal{T}} v^{k+1} + f^k_u (r^k_{uu})^{-1} r^k_u) + (f^k_x)^{\textnormal{T}} v^{k+1} + q^k_x
            \end{multline*}
            с граничными условиями
            \begin{equation*}
                    S_{N+1} = q^{N+1}_{xx},
                    \quad
                    v^{N+1} = q^{N+1}_{x}.
            \end{equation*}
        \end{block}
    \end{frame}
    
    \begin{frame}{Пример: Целевое состояние}
        $$
            q_{\textnormal{терм.}}(x) = \| x - x^{\textnormal{final}}\|^2
        $$
        \begin{figure}
            \centering{
                \includegraphics[width=0.45\textwidth]{img/iLQR/pendulum.pdf}
                \includegraphics[width=0.45\textwidth]{img/iLQR/endpoint.pdf}
            }
            \caption{Траектории руки и схвата для задачи при нулевом начальном управлении $\bar u \equiv 0$.}
        \end{figure}
    \end{frame}


    \section{Выбор начальной траектории алгоритма}

    \begin{frame}{Выбор начальной траектории}
        \begin{block}{Мотивация}
            Предложенный метод сходится медленно. Построим начальное референсное управление $\bar u$ так, чтобы:
            \begin{itemize}
                \item Оно удовлетворяло терминальному условию;
                \item Оно \textit{быстро} строилось;
                \item Оно было допустимо для рассматриваемой задачи.
            \end{itemize}
        \end{block}
    \end{frame}

    \begin{frame}{Референсная траектория}
        Приведем систему к линейной, заменой управления
        $$
            v = M^{-1}(x_1)(\tau - L(x_1, x_2))
        $$
        Тогда система примет вид:
        \begin{equation}\label{eq:linear}
            x^{k+1} =  A x^{k} + B v^{k}
        \end{equation}

        Выберем $x^{\textnormal{final}} \in \textnormal{Argmin}_{x} q^{N+1}(x)$ и решим задачу минимизации

        \begin{equation}\label{eq:linear-cost}
            J = \| x-x^{\textnormal{final}} \|^2 + w \sum_{k=1}^{N} \| v^k \|^2 \longrightarrow \textnormal{min}
        \end{equation}

        Получим референсное управление $u$ из соотношения
        $$
            \tau^k = M(x_1^k)v^k_* + L(x_1^k,x_2^k) \;\Longrightarrow\; u^{k} = \frac{\tau^{k} - \tau^{k-1}}{\Delta t}
        $$
    \end{frame}

    \begin{frame}{Референсная траектория}
        Метод динамического программирования даёт решение задачи.
        \vfill
        \begin{block}{Утверждение}
            Оптимальное управление $v_*$ задачи \eqref{eq:linear}-\eqref{eq:linear-cost} высчитывается как
            $$
                v^k_* = -[wI + B^{\textnormal{T}}P^kB]^{-1} P^k B A x,
            $$
            где матрица $P^k$ может быть посчитана в обратном времени как решение уравнения Риккати
            $$
                \begin{aligned}
                    &P^{k-1} = A^{\textnormal{T}} P^{k} A - A^{\textnormal{T}} P^k B [wI + B^{\textnormal{T}} P^k B]^{-1} B^{\textnormal{T}} P^{k} A
                    \\
                    &P^{N+1} = I
                \end{aligned}
            $$
        \end{block}
    \end{frame}

    \begin{frame}{Пример: целевое состояние}
        $$
            q_{\textnormal{терм.}}(x) = \| x - x^{\textnormal{final}}\|^2
        $$
        \begin{figure}
            \centering{
                \includegraphics[width=0.45\textwidth]{img/initial/pendulum.pdf}
                \includegraphics[width=0.45\textwidth]{img/initial/endpoint.pdf}
            }
            \caption{Траектории руки и схвата для задачи при новом начальном управлении.}
        \end{figure}
    \end{frame}

    \begin{frame}{Пример: целевое состояние}
        $$
            q_{\textnormal{терм.}}(x) = \| x - x^{\textnormal{final}}\|^2
        $$
        \begin{figure}
            \centering{
                \includegraphics[width=0.6\textwidth]{img/initial/compare}
            }
            \caption{Сравнение скорости сходимости алгоритма в зависимости от начальной траектории.}
        \end{figure}
    \end{frame}


    \section{Применение в классических задачах}

    \begin{frame}{Задача: целевое положение схвата}
        $$
            q_{\textnormal{терм.}}(x) = \| e_{\textnormal{схват}}(x) - e^{\textnormal{final}}\|^2
        $$
        \begin{figure}
            \centering{
                \includegraphics[width=0.45\textwidth]{img/examples/reaching_pendulum}
                \includegraphics[width=0.45\textwidth]{img/examples/reaching_endpoint}
            }
            \caption{Траектории руки и схвата для задачи.}
        \end{figure}
    \end{frame}

    \begin{frame}{Задача: целевое положение и скорость схвата}
        $$
            q_{\textnormal{терм.}}(x)
            =
            \| e_{\textnormal{схват}}(x) - e^{\textnormal{final}}\|^2
            +
            w \| \dot e_{\textnormal{схват}}(x) - \dot e^{\textnormal{final}}\|^2
        $$
        \begin{figure}
            \centering{
                \includegraphics[width=0.45\textwidth]{img/examples/reaching_speed_pendulum}
                \includegraphics[width=0.45\textwidth]{img/examples/reaching_speed_endpoint}
            }
            \caption{Траектории руки и схвата для задачи.}
        \end{figure}
    \end{frame}

    \begin{frame}{Задача: обход препятствия}
        $$
            \begin{aligned}
            & q_{\textnormal{терм.}}(x) = \| x - x^{\textnormal{final}}\|^2
            \\
            & q_{\textnormal{фаз.}}(x) = \left(\| e_{\textnormal{cхват}}(x) - e_{\textnormal{препят}}\|^2 - r_{\textnormal{препят}}\right)^2
            \end{aligned}
        $$
        \begin{figure}
            \centering{
                \includegraphics[width=0.45\textwidth]{img/examples/obstacle_pendulum_less}
                \includegraphics[width=0.45\textwidth]{img/examples/obstacle_endpoint_less}
            }
            \caption{Траектории руки и схвата для задачи.}
        \end{figure}
    \end{frame}

    \section{Литература}
    \begin{frame}{Доклады}
        \begin{itemize}
            \item Егоров К. Ю., Востриков И. В. Математическое моделирование движения руки и поведенческих движений // Доклад науч. конф. <<Ломоносовские чтения>> (Москва, 4-14 апреля 2023).
        \end{itemize}
    \end{frame}

    \begin{frame}{Список литературы}
        \begin{enumerate}
            \item[1] Колюбин~С.\,А. Динамика робототехнических систем~// Учебное пособие. --- СПб.: Университет ИТМО, 2017.~--- 117 с.
            \item[2] E. Todorov, M. Jordan. Optimal feedback control as a theory of motor coordination~// Nature Neuroscience, Vol.5, No.11, 1226-1235, 2002.
            \item[3] Y. Uno, M. Kawato, R. Suzuki. Formation and control of optimal trajectory in human multijoint arm movement~--- minimum torque-change model~// Biological Cybernetics 61, 89-101, 1989.
            \item[4] B.D.O. Anderson, J.B. Moore. Optimal Control: Linear Quadratic Methods~// Prentice Hall, Upper Saddle River, 1990.
            \item[5] D. H. Jacobson. Differential dynamic programming methods for determining optimal control of non-linear systems~// University of London, 1967.
        \end{enumerate}
    \end{frame}

    \begin{frame}{Список литературы}
        \begin{enumerate}
            \item[6] E. Guechi, S. Bouzoualegh, Y. Zennir, S. Blažič. MPC Control and LQ Optimal Control of A Two-Link Robot Arm: A Comparative Study~// Machines 6, no. 3: 37, 2018.
            \item[7] A. Babazadeh, N. Sadati. Optimal control of multiple-arm robotic systems using gradient method~// IEEE Conference on Robotics, Automation and Mechatronics, Singapore, pp. 312-317 vol.1, 2004.
        \end{enumerate}
    \end{frame}
\end{document}