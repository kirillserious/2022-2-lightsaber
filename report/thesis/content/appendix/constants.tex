\documentclass[../../doc.tex]{subfiles}
\begin{document}
    \section{Таблица параметров}

    Для возможности сопоставления результатов в работе использовались общие параметры при построении численных решений задач.
    Ниже приведена таблица параметров, которые использовались для построения графиков в случае, если в подписи к соответствующему рисунку не оговорено обратное.

    \begin{flushleft}
    \begin{tabular}{|p{2cm}|p{9cm}|p{4cm}|}
        \hline
        Cимвол & Краткое описание & Значение \\
        \hline\hline
        \multicolumn{3}{|c|}{Параметры модели} \\
        \hline\hline
        $l$ & Длины сочленений & $[0,\!7,\, 0,\!7,\, 1,\!6]$
        \\
        \hline
        $m$ & Массы сочленений & $[0,\!8,\, 0,\!8,\, 1,\!2]$
        \\
        \hline
        $g$ & Ускорение свободного падения & $9,\!8$
        \\
        \hline
        \hline
        \multicolumn{3}{|c|}{Параметры задачи} \\
        \hline\hline
        $t_{\textnormal{start}}$ & Время начала движения & $0$
        \\ \hline
        $t_{\textnormal{final}}$ & Время окончания движения & $1$
        \\ \hline
        $\Delta t$ & Шаг дискретизации & $10^{-3}$
        \\ \hline
        $\theta^{\textnormal{start}}$ & Начальные углы & $[-1,\!4, -1,\!4, -1,\!4]$
        \\ \hline
        $\dot \theta^{\textnormal{start}}$ & Начальные угловые скорости & $[0,\, 0,\, 0]$
        \\ \hline
        $\tau^{\textnormal{start}}$ & Начальные моменты силы & $[0,\, 0,\, 0]$
        \\ \hline
        $w_2$ & Вес энергетического критерия & $10^{-2}$
        \\ \hline\hline
        \multicolumn{3}{|c|}{Параметры основного алгоритма}
        \\\hline\hline
        $\varepsilon$ & Критерий остановки & $10^{-2}$
        \\ \hline
        $\mu$ & Константа регуляризации матрицы $K$ & $10^{-8}$
        \\ \hline
        $\xi_1, \xi_2$ & Интервал регуляризации поправки & $10^{-1}, 10$
        \\ \hline\hline
        \multicolumn{3}{|c|}{Параметры алгоритма поиска начальной траектории}
        \\ \hline\hline
        $\hat w_1$ & Вес фазового критерия & $10^{-2}$
        \\ \hline
        $\hat w_2$ & Вес критерия минимизации ускорения & $10^{-2}$
        \\ \hline
    \end{tabular}
    \end{flushleft}

    \begin{flushleft}\begin{tabular}{|p{2cm}|p{9cm}|p{4cm}|}
        \hline
        \multicolumn{3}{|c|}{Задача перехода в целевое состояние (Разделы 4 и 5)}
        \\ \hline
        $\theta^{\textnormal{final}}$ & Целевые углы & $??$
        \\ \hline
        $\dot \theta^{\textnormal{final}}$ & Целевые угловые скорости & $??$
        \\ \hline
        $\tau^{\textnormal{final}}$ & Целевые моменты силы & $??$
        \\ \hline
        $w_1$ & Вес фазового критерия & $??$
        \\ \hline
    \end{tabular}\end{flushleft}
    \clearpage
\end{document}