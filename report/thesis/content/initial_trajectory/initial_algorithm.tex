\documentclass[../../doc.tex]{subfiles}
\graphicspath{{\subfix{../../img}}}
\begin{document}
    \subsection{Алгоритм}

    Алгоритм \ref{alg:initial:algo} резюмирует метод, предложенный в данном разделе.
    На Рис.~\ref{fig:initial:pendulum} представлен результат работы основного алгоритма из Раздела 4 для задачи \eqref{eq:ilqr-algo:cost} с построенным начальным управлением по предложенному алгоритму.

    \begin{remark}
        Из численного построения управления видно,
        что для задачи \eqref{eq:ilqr-algo:cost}
        при решении с заданной случайной начальной референсной траекторией и траекторией, построенному по методу из данного раздела
        значения функционалов качества $J$ отличаются меньше, чем на заданную константу критерия остановки алгоритма $\varepsilon$, используемую
        при построении управлений
        $$
            | 0,\!456018 - 0,\!456012 | < 0,\!01. 
        $$
    \end{remark}

    Как будет видно на примерах ниже, построение управления при предложенном начальном управлении
    не уводит траекторию из множества, минимизирующего терминальную часть функционала качества,
    который является наиболее важным критерием для целевого движения.
    
    \RestyleAlgo{ruled}
    \begin{rusalgorithm}\caption{Поиск начальной траектории}\label{alg:initial:algo}
        \DontPrintSemicolon
        \SetKwProg{Function}{function}{\;begin}{end function}
        \Function{InitialControl($x^{\textnormal{final}}$)}{
            \tcc{Обратный проход}
            $S_{N+1}, v^{N+1} \gets $ \eqref{eq:initial-S-v-boundary-condition}\;
            \For{$k \gets N$ \KwTo $1$}{
                $S_k, v^{k} \gets $ \eqref{eq:initial-S-v-result}\;
            }
            \;
            \tcc{Прямой проход}
            $\hat x^0 \gets \theta^{\textnormal{start}}, \dot \theta^{\textnormal{start}}$\;
            \For{$k \gets 1$ \KwTo $N$}{
                $\hat u^k, \hat x^{k+1} \gets $ \eqref{eq:initial-optimal-control-result}, \eqref{eq:initial-cauchy-task}\;
            }
            \;
            \tcc{Конвертация управления}
            \For{$k \gets 1$ \KwTo $N$}{
                $u^k \gets $ \eqref{eq:initial-control-translate}
            }
            \Return{u}
        }
        
    \end{rusalgorithm}

    \begin{figure}[h]
        \begin{center}
            \includegraphics[width=0.49\textwidth]{initial/pendulum.pdf}
            \includegraphics[width=0.49\textwidth]{initial/endpoint.pdf}
        \end{center}
        \caption{
            Оптимальная траектория и траектории схвата на различных итерациях алгоритма при решении задачи \eqref{eq:ilqr-algo:cost} с начальным управлением, построеным методом из данного раздела.
            Алгоритм сошелся на $4$ итерации.
            Значение функционала качества $J_*$ на последней итерации равно $0,\!456018$.
        }
        \label{fig:initial:pendulum}
    \end{figure}
    \begin{figure}[h]
        \begin{center}
            \includegraphics[width=0.7\textwidth]{initial/compare.pdf}
        \end{center}
        \caption{Сравнение скорости сходимости для задачи \eqref{eq:ilqr-algo:cost} в зависимости от выбора начального референсного управления.}
        \label{fig:compare-init}
    \end{figure}

    \ifSubfilesClassLoaded{
        \nocite{*}
        \clearpage
        \bibliographystyle{plain}
        \bibliography{../../refs}
    }{}
\end{document}