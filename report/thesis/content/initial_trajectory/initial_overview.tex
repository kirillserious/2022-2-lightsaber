\documentclass[../../doc.tex]{subfiles}

\begin{document}
    \subsection{Описание метода}

    Скорость сходимости метода, приведенного в Разделе~4, зависит от выбора начального референсного управления $\bar u$.
    Считается, что в этом деле можно положиться на мнение эксперта в предметной области, который нарисует траекторию системы $\bar x$ и мы потом решим задачу идентификации управления $\bar u$, приводящего к такой траектории методом следящего управления.

    Для того, чтобы использовать метод необходимо, чтобы он:
    \begin{enumerate}\itemsep0em
        \item Строил управление \textit{быстро}. Желательно, чтобы алгоритм имел линейную асимптотику.
        \item Получившаяся референсная траектория была близка к оптимальной.
        \item Получившаяся референсная траектория была бы возможной для рассматриваемой задачи. Это важно, если в задаче присутствуют cтрогие ограничения на управление.
    \end{enumerate}
    
    В отсутствии эксперта, предлагается использовать следующий метод, удовлетворяющий этим условиям.
    \begin{enumerate}
        \item Необходимо аналитически найти состояние системы, которое минимизирует терминальное условие
        $$
            x^{\textnormal{final}} \in \textnormal{Argmin } q^{\textnormal{final}}(x).
        $$
        \item Привести систему к линейной и поставить для нее задачу минимизации интегрально-квадратичного функционала для перехода в состояние $x^{\textnormal{final}}$.
        \item Построить линейно-квадратичный регулятор для полученной задачи. Тем самым мы получим управление, которое минимизирует терминальное условие, но ничего не говорит об энергетическом и фазовом условиях. Тем не менее такой подход будет работать лучше, чем выбор случайного управления.
    \end{enumerate}
    В случае, если мы можем аналитически найти несколько минимизаторов терминального условия, можно провести перебор с последующим выбором самого подходящего управления.

    \ifSubfilesClassLoaded{
        \nocite{*}
        \clearpage
        \bibliographystyle{plain}
        \bibliography{../../refs}
    }{}
\end{document}



