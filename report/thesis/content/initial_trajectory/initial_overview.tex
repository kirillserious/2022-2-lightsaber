\documentclass[../../doc.tex]{subfiles}

\begin{document}
    \subsection{Описание метода}

    Скорость сходимости метода, приведенного в Разделе~4,
        зависит от выбора начального референсного управления $\bar u$.
    Считается, что при выборе начального управления
        можно положиться на мнение эксперта в предметной области,
        который предложит траекторию системы $\bar x$, близкую к оптимальной.
    В этом случае перед применением метода
        необходимо решить задачу оптимального преследования,
        то есть найти управление $\bar u$, приводящее систему к заданной траектории $\bar x$.
    
    Однако в отсутствии эксперта необходимо предложить метод для построения начальной референсной траектории.
    Выпишем основные критерии, которым должен удовлетворять медод:

    \begin{enumerate}\itemsep0em
        \item Метод должен строить управление \textit{быстро} --- желательно, чтобы алгоритм имел линейную асимптотику;
        \item Получившаяся референсная траектория должна быть близкой к оптимальной;
        \item Получившаяся референсная траектория должна быть возможной для рассматриваемой задачи. Данное условие важно, если в задаче присутствуют ограничения на управление.
    \end{enumerate}
    
    В данном разделе предложен метод удовлетворяющий критериям выше. Выпишем его основные шаги:
    \begin{enumerate}
        \item Необходимо аналитически найти состояние системы, которое минимизирует терминальное условие
        $$
            x^{\textnormal{final}} \in \textnormal{Argmin } q^{\textnormal{final}}(x);
        $$
        \item Привести систему к линейной и поставить для нее задачу минимизации интегрально-квадратичного функционала для перехода в состояние~$x^{\textnormal{final}}$;
        \item Построить линейно-квадратичный регулятор для полученной задачи.
    \end{enumerate}
    В случае, если мы можем аналитически найти несколько минимизаторов терминального условия,
    можно провести перебор: построить начальную траекторию для каждого минимизатора,
    а затем выбрать управление с минимальной величиной функционала качества~$J$.

    Тем самым мы получим управление, которое минимизирует терминальное условие, но ничего не говорит об энергетическом и фазовом условиях.
    Тем не менее такой подход будет работать лучше, чем выбор случайного управления.
    Сравнение скорости сходимости для задачи~\eqref{eq:ilqr-algo:cost} для нулевого начального управления и начального управления, предложенного в данном разделе, приведено на Рис.~\ref{fig:compare-init}.

    \begin{remark}
        Как мы увидим далее, полученная референсная траектория минимизирует угловые ускорения сочленений руки.
        Таким образом результирующая траектория будет самой плавной из возможных.
        С физической точки зрения такая траектория кажется допустимой для руки,
        что позволяет использовать её и в задачах с наличием ограничений на управление.
    \end{remark}

    \ifSubfilesClassLoaded{
        \nocite{*}
        \clearpage
        \bibliographystyle{plain}
        \bibliography{../../refs}
    }{}
\end{document}



