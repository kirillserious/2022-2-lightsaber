\documentclass[../../doc.tex]{subfiles}

\begin{document}
    \subsection{Синтез управления}

    Приведём систему \eqref{eq:discrete-system} к линейному виду заменой управления на
    \begin{equation}\label{eq:initial-control-definition}
        \hat u = M^{-1}(x_1)[\tau - L(x_1, x_2)].
    \end{equation}
    Тогда в фазовом пространстве $\hat x = [x_1, x_2]\T \in \mathbb{R}^{6}$ задача~Коши примет вид
    \begin{equation}\label{eq:initial-cauchy-task}
        \begin{cases}
            \hat x^{k+1} = \hat A \hat x^{k} + \hat B \hat u^{k},\; k=\overline{1,N},
            \\
            \hat x^{1} = x^{\textnormal{start}},
        \end{cases}
    \end{equation}
    где
    \begin{equation*}
        \hat A
        =
        \begin{pmatrix}
            \;I\; & \hspace*{-\arraycolsep}\vline\hspace*{-\arraycolsep} & \Delta t I
            \\
            \hline
            \;O\; & \hspace*{-\arraycolsep}\vline\hspace*{-\arraycolsep} & I
        \end{pmatrix}
        ,\quad
        \hat B
        =
        \begin{pmatrix}
            O
            \\
            \hline
            \Delta t I
        \end{pmatrix}
        .
    \end{equation*}

    Будем считать, что для исходной задачи задачи~\eqref{eq:discrete-system}-\eqref{eq:discrete-cost} известно некоторое состояние $x^{\textnormal{final}}$, минимизирующее терминальное условие $q^{\textnormal{final}}$, то есть
    \begin{equation*}\label{eq:initial-x-final-definition}
        x^{\textnormal{final}} \in \textnormal{Argmin}\; q^{\textnormal{final}}(x).
    \end{equation*}
    В таком случае для задачи~Коши~\eqref{eq:initial-cauchy-task} поставим задачу минимизации следующего функционала
    \begin{equation}\label{eq:initial-cost-definition}
        J
        =
            \left\| \hat x^{N+1} - x^{\textnormal{final}} \right\|^2
            +
            \hat w_1 \sum_{k=1}^{N} \left\langle \hat x^k, \hat Q \hat x^k \right\rangle
            +
            \hat w_2 \sum_{k=1}^{N} \left\| \hat u^k \right\|^2
        \longrightarrow
        \textnormal{min}.
    \end{equation}
    
    Здесь матрица $\hat Q = \hat Q\T$ выбирается для исключения возможного \textit{проворачивания} сочленений относительно друг друга следующим образом
    \begin{equation}\label{eq:initial-phase-definition}
        \hat Q
        =
        \begin{pmatrix}
                \begin{matrix}
                    \hphantom{-}2 & -1            & \hphantom{-}0 \\
                    -1            & \hphantom{-}2 & -1            \\
                    \hphantom{-}0 & -1            & \hphantom{-}1
                \end{matrix}
                & \hspace*{-\arraycolsep}\vline\hspace*{-\arraycolsep} &
                O
            \\\hline
                O
                & \hspace*{-\arraycolsep}\vline\hspace*{-\arraycolsep} &
                O
        \end{pmatrix}
        .
    \end{equation}
    Иными словами матрица $\hat Q$ является матрицей следующей квадратичной формы
    \begin{equation*}
        \left\langle
            \hat x, \hat Q \hat x
        \right\rangle
        =
        \theta_1^2 + (\theta_2 - \theta_1)^2 + (\theta_3 - \theta_2)^2.
    \end{equation*}
    Данное фазовое условие штрафует траекторию в случае большого относительного отклонения между углами сочленений.

    После решения задачи \eqref{eq:initial-cauchy-task}-\eqref{eq:initial-cost-definition} мы можем восстановить соответствующее управление исходной задачи.
    Пусть $\hat u^{*}$, $\hat x^{*}$~--- оптимальное управление и соответствующая ему оптимальная траектория задачи \eqref{eq:initial-cauchy-task}-\eqref{eq:initial-cost-definition}.
    Тогда соответствующее управление для исходной задачи $u$, можно получить по формуле
    \begin{equation}\label{eq:initial-control-translate}
        u^{k} = \frac{\hat \tau^{k} - \hat \tau^{k-1}}{\Delta t},
    \end{equation}
    где
    \begin{equation*}
        \begin{cases}
            \hat \tau^k = M(\hat x_1^{k*})\hat u^{k*} + L(\hat x_1^{k*}, \hat x_2^{k*}), \mbox{ при } k=\overline{1, N},
            \\
            \hat \tau^{0} = \tau^{\textnormal{start}}.
        \end{cases}
    \end{equation*}

    Построим гамильтониан задачи \eqref{eq:initial-cauchy-task}-\eqref{eq:initial-cost-definition}
    \begin{equation}\label{eq:initial-hamiltonian}
        \hat H_k
        =
            \left\langle \hat x^k, \hat w_1 \hat Q \hat x^k \right\rangle
            +
            \left\langle \hat u^k, \hat w_2 \hat u^k \right\rangle
            +
            (\hat \lambda^{k+1})\T [\hat A \hat x^k + \hat B \hat u^k]
        .
    \end{equation}
    Оптимальное управление $\hat u^*$ должно удовлетворять необходимому условию оптимальности:
    \begin{equation*}
            \left.\frac{\partial \hat H_k}{\partial \hat u^k}\right|_{\hat u^k = \hat u^{k*}}
        =
            \hat w_2 \hat u^{k*} + \hat B \T \hat \lambda^{k+1}
        =
            0,
    \end{equation*}
    что дает следующее выражение для управления
    \begin{equation}\label{eq:initial-optimal-control-definition}
        \hat u^{k*} = -\frac{1}{\hat w_2} \hat B \T \hat \lambda^{k+1}.
    \end{equation}
    И уравнение \eqref{eq:initial-cauchy-task} можно переписать в следующем виде:
    \begin{equation}\label{eq:initial-x-over-lambda}
        \hat x^{k+1} = \hat A \hat x^{k} - \frac{1}{\hat w_2} \hat B \hat B\T \hat \lambda^{k+1}.
    \end{equation}
    При этом имеет силу следующая сопряженная система:
    \begin{equation}\label{eq:initial-lambda-system-definition}
        \begin{cases}
            \hat \lambda^{k} = \hat w_1 \hat Q \hat x^{k} + \hat A\T \hat \lambda^{k+1}, \mbox{ при } k = \overline{1, N}
            \\
            \hat \lambda^{N+1} = \hat x^{N+1} - x^{\textnormal{final}}
        \end{cases}
    \end{equation}
    
    \begin{theorem}
        Оптимальное управление $\hat u^*$ задачи \eqref{eq:initial-cauchy-task}-\eqref{eq:initial-cost-definition} задается формулой
        \begin{equation}\label{eq:initial-optimal-control-result}
            \hat u^{k*} = \hat L_{k} \hat x^{k} + \hat d^k,
        \end{equation}
        где
        \begin{equation*}
            \begin{aligned}
                & \hat L_k = -\frac{1}{\hat w_2} \hat B\T S_{k+1}\left( I + \frac{1}{\hat w_2} \hat B \hat B\T S_{k+1} \right)^{-1} \hat A,
                \\
                & \hat d^k = -\frac{1}{\hat w_2} \hat B\T \left( I - S_{k+1} \left( I + \frac{1}{\hat w_2} \hat B \hat B\T S_{k+1} \right)^{-1} \hat B \hat B\T \right) v^{k+1}.
            \end{aligned}
        \end{equation*}
        Причем переменные $S_{k}$, $v^{k}$ могут быть найдены в обратном времени из соотношений
        \begin{equation}\label{eq:initial-S-v-result}
            \begin{aligned}
                &
                        S_k
                    =    
                            \hat w_1 \hat Q
                        +
                            \hat A\T S_{k+1} \left(
                                    I + \frac{1}{\hat w_2} \hat B \hat B\T S_{k+1}     
                            \right)^{-1} \hat A
                    ,
                \\
                &
                        v_k
                    =
                        \hat A\T
                        \left(
                            I - \frac{1}{\hat w_2} S_{k+1} \left(
                                I + \frac{1}{\hat w_2} \hat B \hat B\T S_{k+1}     
                        \right)^{-1} \hat B \hat B\T
                        \right)
                        v^{k+1}
            \end{aligned}
        \end{equation}
        с граничными условиями
        \begin{equation}\label{eq:initial-S-v-boundary-condition}
            \begin{aligned}
                & S_{N+1} = I,
                \\
                & v^{N+1} = - x^{\textnormal{final}}.
            \end{aligned}
        \end{equation}
    \end{theorem}
    
    \begin{proof}
        Аналогично доказательству теоремы~(1) будем искать решение сопряженной системы в аффинном виде
        \begin{equation}\label{eq:initial-lambda-affin}
            \hat \lambda^{k} = S_k \hat x^k + v^k.
        \end{equation}
        Из сопряженной системы~\eqref{eq:initial-lambda-system-definition} получаем граничные условия:
        \begin{equation*}
            \hat \lambda^{N+1} = \hat x^{N+1} - x^{\textnormal{final}}
            \quad\Longrightarrow\quad
            S_{N+1} = I,
            \;
            v^{N+1} = - x^{\textnormal{final}}.
        \end{equation*}
        Подставив выражение~\eqref{eq:initial-lambda-affin} в уравнение \eqref{eq:initial-x-over-lambda}, получим
        \begin{equation*}
            \hat x^{k+1} = \hat A \hat x^{k} - \frac{1}{\hat w_2} \hat B \hat B\T (S_{k+1} \hat x^{k+1} + v^{k+1}),
        \end{equation*}
        откуда выражаем
        \begin{equation*}
            \hat x^{k+1}
            =
                \left(
                    \underbrace{
                        I + \frac{1}{\hat w_2} \hat B \hat B\T S_{k+1}
                    }_{K_k}
                \right)^{-1}
                \left(
                    \hat A \hat x^{k} - \frac{1}{\hat w_2} \hat B \hat B\T v^{k+1}
                \right).
        \end{equation*}
        Теперь подставим получившееся выражение в \eqref{eq:initial-lambda-system-definition}:
        \begin{multline*}
                \hat \lambda^{k}
            =
                S_k \hat x^{k} + v^{k}
            =
                    \hat w_1 \hat Q \hat x^{k}
                +
                    \hat A\T (S_{k+1} \hat x^{k+1} + v^{k+1})
            =\\=
                    \hat w_1 \hat Q \hat x^{k}
                +
                    \hat A\T
                    S_{k+1} 
                    K_k^{-1}
                    \left(
                        \hat A \hat x^{k} - \frac{1}{\hat w_2} \hat B \hat B\T v^{k+1}
                    \right)
                +
                    \hat A\T v^{k+1}
            =\\=
                    \left(
                            \hat w_1 \hat Q
                        +
                            \hat A\T S_{k+1} K_k^{-1} \hat A
                    \right) \hat x^{k}
                +
                    \hat A\T
                    \left(
                        I - \frac{1}{\hat w_2} S_{k+1} K_k^{-1} \hat B \hat B\T
                    \right)
                    v^{k+1}.
        \end{multline*}
        Откуда получаем искомые соотношения:
        \begin{equation*}
            \begin{aligned}
                &
                        S_k
                    =    
                            \hat w_1 \hat Q
                        +
                            \hat A\T S_{k+1} \left(
                                    I + \frac{1}{\hat w_2} \hat B \hat B\T S_{k+1}     
                            \right)^{-1} \hat A
                    ,
                \\
                &
                        v_k
                    =
                        \hat A\T
                        \left(
                            I - \frac{1}{\hat w_2} S_{k+1} \left(
                                I + \frac{1}{\hat w_2} \hat B \hat B\T S_{k+1}     
                        \right)^{-1} \hat B \hat B\T
                        \right)
                        v^{k+1}
                    .
            \end{aligned}
        \end{equation*} 
        Теперь выражение для оптимального управления \eqref{eq:initial-optimal-control-result} получается прямой подстановкой получившихся соотношений в выражение \eqref{eq:initial-optimal-control-definition}.
     
    \end{proof}

    \ifSubfilesClassLoaded{
        \nocite{*}
        \clearpage
        \bibliographystyle{plain}
        \bibliography{../../refs}
    }{}
\end{document}



