\documentclass[../doc.tex]{subfiles}
\graphicspath{{\subfix{../img}}}
\begin{document}
    В рамках данной работы был предложен метод управления биологической системой,
    соответствующий реальному поведению человека при осуществлении целевого движения.

    С целью возможности получения численных результатов разработанного метода,
        применимых к реальным задачам биомоторики, было проведено следующее:
    \begin{enumerate}\itemsep0em
        \item Предложена математическая модель планарной руки человека, держащей предмет,
            и выведено уравнение динамики для данной системы;
        \item Постулирован принцип оптимальности, дающий возможность применения методов
            оптимального управления для построения траекторий движения, соответствующих
            траекториям человека;
        \item Рассмотрены предложенные в литературе формализации энергетического критерия оптимальности, минимизация которого присуща биологическому движению;
        \item В соответствии с полученным уравнением динамики и выбранным критерием оптимальности поставлена задача оптимального целевого управления системой в непрерывной и дискретной формах.
    \end{enumerate}
    
    Для решения поставленной задачи были рассмотрены известные базовые методы решения нелинейных
        задач оптимального управления, и на основе метода итеративного линейно-квадратичного регулятора
        был разработан метод построения оптимального управления для дискретной постановки задачи.
    Особенностью метода, предложенного в данной работе, является способ регуляризации итеративной поправки на управление,
        позволяющий улучшить сходимость базового метода для рассматриваемой задачи.
    Рассмотрены альтернативные варианты регуляризации поправки и возможность их применения для задач
        с ограничениями на управление.
    Для демонстрации работы метода было реализовано программное решение.

    С целью уменьшения числа итераций предложенного алгоритма и, соответственно, времени работы
        программного решения был предложен способ построения начального референсного управления как
        линейно-квадратичного регулятора приведенной к линейному виду задачи для минимизации терминального
        критерия исходной задачи.
    Данный способ быстро работает и не опирается на мнение эксперта в области биологической моторики при построении начальной референсной траектории.
    На примере было показано, что он действительно снижает необходимое число итераций основного алгоритма.
    
    Наконец для возможности сравнения предложенного метода с имеющимися в литературе в качестве примера были
        рассмотрены конкретные постановки задачи, являющиеся классическими задачами биомеханического движения:
    \begin{enumerate}\itemsep0em
        \item Переход в целевое состояние;
        \item Переход в целевое положение схвата;
        \item Переход в целевое положение схвата с заданной скоростью;
        \item Задача обхода препятствия.
    \end{enumerate}
    Для рассмотренных классических задач:
    \begin{enumerate}\itemsep0em
        \item Представлена формализация в подходящем для предложенного метода виде,
            а также способ нахождения минимизатора терминального критерия для построения начальной референсной траектории;
        \item При помощи программного решения построено оптимальное управление, и приведена соответствующая ему траектория системы;
        \item Приведено число итераций, потребовавшихся программному решению для построения оптимального управления.
    \end{enumerate}

    Таким образом, работа содержит полный анализ движения руки человека, держащей предмет: от построения
        математической модели и постановки задачи до разработки метода её решения и его имплементации в виде
        программного решения.
    Основными научными результатами работы являются метод построения оптимального управления и метод построения 
        начальной референсной траектории, выраженными в Теореме 1, Теореме 2 и предложенном способе
        регуляризации оптимальной поправки (Подраздел 4.3).

    \ifSubfilesClassLoaded{
        \nocite{*}
        \clearpage
        \bibliographystyle{plain}
        \bibliography{../../refs}
    }{}
\end{document}