\section{Задача отбивания мяча}

Теперь формализуем отбитие мяча. Пусть в некоторый момент $t$ мы имеем предмет~--- точечное тело массы $m$ и он летит со скоростью $v$ по направлению $x$.
И вот мы его как-то отбиваем. Тут мы будем считать, что тра-та-та мы везем с собой кота, Чижика, собаку, кошку-забияку. И короче у нас есть два терминальных условия:
\begin{enumerate}
    \item Условие на соприкосновение. Тут все просто у меня есть готовая формула
    \item Условие на отбитие куда я хочу. Вот тут типа жопа, и надо выписать закон сохранения чего-то там, и от туда получить что-то еще. Ну в общем все не так плохо.
\end{enumerate}

\subsection{Мяч движется в пространстве}

Мяч теперь движется с некоторым известным уравнением мяча $x(t)$ по траектории. Мы для примеров будем рассматривать движение мяча под углом к горизонту.

Тут я короче параметризую время $t = a\tau$, и система (какая-то) переписывается в виде:
$$
    hard formulae
$$

При этом мы знаем время влета в $\mathcal{B}_0(\sum_{i=1}{3}l_i)$ и вылета из этого множества нам известно.

Мы введем сетку по $a$ и построим много референсных траекторий, каждая из которых будет приводить систему в нужное время, в нужное место.
Затем мы будем параллельно улучшать каждую из этих траекторий.