\section{Задача отбивания мяча}

\subsection{Задача с известным терминальным временем}
Теперь формализуем отбитие мяча. 
Рассмотрим точечное тело массы $m$, летящее в пространстве по некоторому заранее известному закону $\hat x(t)$.
Поставим задачу оптимального отбивания этого мяча.

Допустим, что нам известно оптимальное время отбития $t_1$.
Пусть в терминальный момент времени $t_1$ мяч находится в точке $e^* = \hat x(t_1)$.
В таком случае нам необходимо формализовать два терминальных условия:
\begin{enumerate}
    \item Условие на соприкосновение третьего сочленения системы с мячом,
    \item Условие на необходимую приобретённую мячом скорость, после соприкосновения.
\end{enumerate}

Рассмотрим эти условия по-отдельности.
Формализация первого условия следует из геометрических свойств задачи:
\begin{equation}
    J_{\textnormal{сопр.}} = \|e^2 - e^* \| + \| e^3 - e^* \|.
\end{equation}

Для формализации второго условия выпишем закон сохранения импульса для системы <<третье сочлеление--мяч>>:
{\color{red} Выписать таки это дурацкое уравнение, которое никак не выходит.}
Таким образом, в предположении что {\color{red} что-то там} получаем следующее условие:
\begin{equation}
    J_{\textnormal{скор.}} = \| \dot e - \dot e^{\textnormal{target}} \|^2,
\end{equation}
где $e$~--- координата точки, лежащей на третьем сочленении, удаленной от точки соединения на $\| e^2 - e^* \|$.

Причем первый критерий должен считаться более приоритетным, то есть
\begin{equation*}
    w_{\textnormal{сопр.}} > w_{\textnormal{скор.}}.
\end{equation*}

Для построения референсной траектории выбирается терминальное положение таким образом, чтобы центр третьего сочленения касался мяча с заданной скоростью, то есть
$$
    \textnormal{формула для } \theta = \theta(e^*) \textnormal{ через много синусов}
$$

\subsection{Задача с неизвестным терминальным временем}

К сожалению общего метода для решения задач с неизвестным концом нет.
В работе [ссылка] рассматривается вариант без интегрального условия.
В работах [ссылка], [ссылка]~--- строются итеративные поправки на конечное время в предположении, что терминальное условие обращается в ноль конечное число раз.

В нашем же случае ничего не остается, кроме как устроить полный перебор терминального времени, и решения предыдущей задачи заново.
Тем не менее данный подход не лишен достоинств: его можно легко распараллелить.

Пусть $\hat t_0 < \hat t_1 \in \{t | x(t) \cap \mathcal{B}_0(\sum_{i=1}^{3}l_i) \}$.
Таких точек действительно две, если мы рассматриваем движение мяча, как тела, летящего под углом к горизонту.
Тогда получим $M$ задач, каждую из которых мы решаем независимо.