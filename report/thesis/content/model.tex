\section{Математическое моделирование}

\subsection{Планарная модель руки человека}

Рассмотрим руку человека, держащего стержень.
В некотором приближении можно считать, что мы имеем трехсекционный математический маятник.
Для каждого из $3$-x сочленений нам известны:
\begin{enumerate}
    \item Масса сочленения $m_i$, $i=1,2,3$;
    \item Линейная плотность сочленения $\rho_i = \rho_i(x)$, $0 \leqslant x \leqslant l_i$, $i=1,2,3$;
    \item Длина сочленения $l_i$, $i=1,2,3$;
    \item Угол поворота сочленения $\theta_i$, $i=1,2,3$ относительно оси абсцисс $Oe_1$.
\end{enumerate}

Также считаем, что положение плечевого сустава фиксировано для определенности в точке $O = (0,0)$.

\subsection{Уравнение динамики системы}

Для получения уравнения динамики рассматриваемой физической системы воспользуемся методом Эйлера--Лагранжа [ссылка].
Идея метода состоит в проведении следующих последовательных шагов:
\begin{enumerate}
    \item Выбор обобщенных координат;
    \item Получение выражения для кинетической $K$ и потенциальной $\Pi$ энергий системы, записанных в обобщенных координатах;
    \item Получение выражения для лагражиана системы $\mathcal{L}$;
    \item Составление системы уравнений движения, соответствующих каждой обобщенной координате.
\end{enumerate}

Обобщенными координатами для нашей системы выберем углы поворота сочленений $\theta_i$, $i = 1, 2, 3$.
Далее перейдем к выражению энергий через обобщенные координаты.

Для подсчета кинетической энергии воспользуемся теоремой Кёнинга~[ссылка куда-то].
\begin{theorem}[Кёнинг]
    Кинетическая энергия тела есть энергия поступательного движения центра масс плюс энергия вращательного движения относительно центра масс
    \begin{equation}\label{eq:kin-energy}
        K = \frac{1}{2} m \|v_c\|^2 + \frac{1}{2}\omega^{\T} I \omega,
    \end{equation}
    где $m$~--- полная масса тела, $I$~--- тензор инерции тела, $v_c$~--- линейная скорость центра масс, $\omega$~--- скорость вращения тела относительно центра масс.
\end{theorem}

Далее в работе мы будем полагать, что каждое из сочленений представляет собой однородный стержень длины $l_i$ массы $m_i$.
В таком случае получаем следующие значения для положения центра масс~$c^i \in \mathbb{R}^2$ $i$-ого сочленения:
$$
    c^i =
    \left[\begin{aligned}
        \sum_{j=1}^{i-1} l_j cos\theta_j + \frac{l_i}{2}cos\theta_i \\
        \sum_{j=1}^{i-1} l_j sin\theta_j + \frac{l_i}{2}sin\theta_i
    \end{aligned}\right].
$$
Выражения для момента инерции и скорости вращательного движения относительно центра масс для стержня получаются соответственно:
$$
    I_i = \uint\limits_{(m_i)} r^2\,dm = \rho_i\int\limits_{(l_i)} r^2\,dl = \frac{m_il_i^2}{12},
$$
$$
    \omega_i = 2\dot \theta_i.
$$
Потенциальная энергия $i$-ого сочленения рассчитывается по формуле
$$
    \Pi_i = m_i g c^i_2,
$$
где $g \approx 9,\!8$~--- ускорение свободного падения на поверхности Земли.

Общая кинетическая и потенциальная энергии системы рассчитываются как сумма энергий каждого из сочленений:
$$
    K = \sum_{i=1}^{3} K_i = \sum_{i=1}^{3}\left( \frac{m_i \|\dot c^i\|^2}{2} + \frac{m_i l_i^2 |\dot \theta_i|^2}{6} \right),
$$
$$
    \Pi = \sum_{i=1}^{3} \Pi_i = \sum_{i=1}^{3} m_i g c^i_2.
$$

Теперь введём лагражиан системы
$$
    \mathcal{L} = K - \Pi
$$
и построим систему уравнений Эйлера--Лагранжа~[ссылка]:
\begin{equation}\label{eq:euler-lagrange}
    \frac{d}{dt} \frac{\partial \mathcal{L}}{\partial \dot \theta_i} - \frac{\partial \mathcal{L}}{\partial \theta_i} = \tau_i,\; i = \overline{1, 3},
\end{equation}
где $\tau_i$~--- момент силы, действующий на $i$-ое сочленение, который доступен для управления.

Продифференцировав члены из левой части уравнения \eqref{eq:euler-lagrange}, получим уравнение динамики для рассматриваемой системы:
\begin{equation}\label{eq:dynamic}
    M(\theta)\ddot\theta + L(\theta, \dot\theta) = \tau,
\end{equation}
где $M(\theta) = M^{\T}(\theta) \in \mathbb{R}^{3 \times 3} > 0$~--- матрица инерции системы, $L(\theta, \dot\theta)\in\mathbb{R}^{3}$~--- вектор центростремительных и кореолисовых сил.


\subsection{Учёт энергетических затрат}

Биологическое движение требует обработки большого количества информации.

{\color{red} Cюда можно добавить воды.}

Моторно-двигательная система как результат эволюции и обучения строит движение в соотвествии с принципом оптимальности~[ссылка].

\begin{assertion}[Биологический принцип оптимальности]
    Выбираемые нервной системой схемы движения являются оптимальными для поставленной задачи.
\end{assertion}

В работе [ссылка] было показано, что оптимизации проводятся с целью уменьшения затрат энергии.
Однако общего подхода к формализации энергетических затрат пока не выработано.
Так, например, в работе~[ссылка] предлагается минимизировать \textit{рывок} схвата, то есть
$$
    J = \uint\limits_{t_0}^{t_1}\left\|
        \frac{d^3e^3}{dt^3}
    \right\|^2\, dt \longrightarrow \mathrm{min},
$$
а в работе~[ссылка]~--- изменение крутящего момента:
\begin{equation}\label{eq:energy-cost}
    J = \uint\limits_{t_0}^{t_1}\left\|
        \frac{d\tau}{dt}
    \right\|^2\, dt \longrightarrow \mathrm{min}.
\end{equation}
Причём, существуют и другие менее популярные варианты~[ссылка].

Мы будем использовать для формализации энергетических затрат выражение~\eqref{eq:energy-cost}, так как оно напрямую зависит от динамики руки и, вероятно, лучше предсказывает траектории биологического движения.

\subsection{Уравнение кинематики системы}

Рассмотрим расширенное фазовое пространство.
Введем состояние системы как
$$
    x = [\theta\;\dot\theta\;\tau]^{\T} \in \mathbb{R}^7.
$$
Тогда уравнение динамики системы~\eqref{eq:dynamic} можно переписать в виде системы однородных дифференциальных уравнений
\begin{equation}\label{eq:kinematic}
    \dot x = A(x) + Bu,
\end{equation}
где 
$$
    A(x) = \left[\begin{aligned}
        x_2 \\
        M^{-1}(x_1)(x_3 - L(x_1, x_2)) \\
        0
    \end{aligned}\right]
    ,\quad
    B = \left[\begin{aligned}
        0 \\
        0 \\
        1
    \end{aligned}\right].
$$
Считаем, что для данной системы поставлена задача~Коши, то есть нам известно начальное состояние системы:
\begin{equation}\label{eq:cauchy}
    x(t_0) = x^0.
\end{equation}
\begin{remark}
    Отметим, что получившаяся функция $A(x)$ бесконечное число раз непрерывно дифференцируема всюду.
\end{remark}
\begin{remark}
    Отметим также, что для выполнения достаточных условий существования и единственности решения Каратеодори для задачи Коши~\eqref{eq:kinematic}-\eqref{eq:cauchy} управление~$u(\cdot)$ достаточно брать из класса измеримых на рассматриваемом отрезке $t_0\leqslant t \leqslant t_1$ функций.
\end{remark}

Для задачи Коши~\eqref{eq:kinematic}-\eqref{eq:cauchy} поставим задачу поиска управления $u(\cdot) \in \mathcal{U} \subset U[t_0,t_1]$, минимизирующего функционал вида:
\begin{equation}\label{eq:continuos-cost}
    J = q^{final}(x(t_1)) + w_1\int\limits_{t_0}^{t_1} q(x(t))\,dt + w_2\int\limits_{t_0}^{t_1} r(u(t))\,dt,
\end{equation}
где $q^{final}(\cdot)$, $q(\cdot)$ отвечают за терминальное и фазовые ограничения соответственно и выбираются в зависимости от конкретной постановки задачи, а $r(\cdot)$ отвечает за энергетические затраты и в соответствии с \eqref{eq:energy-cost} равна:
$$
    r(u) = \|u\|^2,
$$
а $w_1$, $w_2$~--- веса соответствующих критериев для данной многокритериальной задачи.


\subsection{Дискретизация задачи}

{\color{red} Тут уже чувствуется, что буквы начали повторяться.}

Для удобства дальнейших рассуждений дискретизируем задачу~\eqref{eq:kinematic}-\eqref{eq:cauchy}-\eqref{eq:continuos-cost} по времени $t_0 \leqslant t \leqslant t_1$.
Для этого введем равномерную сетку с шагом~$\Delta t$:
$$
    \{ t_i \}_{i=0}^{N+1}, \quad t_0 = t_0, \quad t_N = t_1, \quad t_{i+1} - t_{i} = \Delta t.
$$
Тогда, сузив класс допустимых управлений до кусочно-постоянных, получаем дискретный вариант рассматриваемой задачи Коши~\eqref{eq:kinematic}-\eqref{eq:cauchy}:
\begin{equation}\label{eq:discrete-system}
    \left\{\begin{aligned}
        &x^{k+1} = f(x^k, u^k), \\
        &x^{0} = x^{0},
    \end{aligned}\right.
\end{equation}
где $f(x^k, u^k) = \Delta t (A(x^k) + B u^k) + x^k$.

При этом функционал \eqref{eq:continuos-cost} для дискретной задачи приобретет вид
\begin{equation}\label{eq:discrete-cost}
    J = q^{N+1}(x^{N+1}) + w_1 \sum_{i=0}^{N} \Delta t q(x^N) + w_2 \sum_{i=0}^{N} \Delta t r(u^N).
\end{equation}
