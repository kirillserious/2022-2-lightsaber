\documentclass[../../doc.tex]{subfiles}

\begin{document}
    \subsection{Описание метода}

    Есть два базовых метода для решения задач типа \eqref{eq:discrete-system}-\eqref{eq:discrete-cost}:
    \begin{enumerate}\itemsep0em
        \item Метод дифференциального динамического программирования (DDP) \cite{mayne1966}, \cite{murray1984};
        \item Метод итеративного линейно-квадратичного регулятора (iLQR) \cite{li2004}.
    \end{enumerate}
    Оба метода итеративны и требуют на каждой итерации некоторое \textit{референсное} управление $\bar u$ и соответствующую ему референсную траекторию $\bar x$.
    Далее в работе под референсной траекторией понимается пара $(\bar u, \bar x)$.
    Вдоль данной траектории задача Коши и функционал качества полиномиально аппроксимируются.
    После чего к аппроксимированной системе применяется соответствующий метод.
    Результатом итерации является поправка на референсное управление $\delta u$.
    
    Метод DDP cтроит поправку как градиент гамильтониана аппроксимированной задачи
    $$
        \delta u^k = \alpha \nabla_u H(\bar u^k),
    $$
    метод iLQR~--- как линейно-квадратичный регулятор.

    Считается, что метод iLQR более надежный, так как не подвержен проблемам, присущим градиентным методам, таким как остановка в локальном минимуме,
    но сходится за большее число итераций, чем метод DDP.
    Однако при проведении сравнения скорости сходимости на конкретных примерах выясняется, что нельзя заранее предсказать, какой метод покажет себя лучше~\cite{manchester2016}.

    В данной работе применяется метод iLQR. Его основная идея:
    \begin{enumerate}\itemsep0em
        \item На каждой итерации имеем референсную траекторию $(\bar u, \bar x)$;
        \item Вдоль референсной траектории линеаризуем задачу Коши и аппроксимируем функционал качества до второго порядка;
        \item Строим поправку на управление $\delta u$ как линейно-квадратичный регулятор аппроксимированной задачи;
        \item Если не выполнено терминальное условие $|J(\bar u) - J(\bar u + \delta u)| < \varepsilon$ используем поправленное управление $\bar u + \delta u$ в качестве референсного на следующей итерации алгоритма.
    \end{enumerate}

    \ifSubfilesClassLoaded{
        \nocite{*}
        \clearpage
        \bibliographystyle{plain}
        \bibliography{../../refs}
    }{}
\end{document}