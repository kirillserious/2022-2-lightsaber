\documentclass[../../doc.tex]{subfiles}

\begin{document}
    \subsection{Синтез оптимальной поправки}
    
    Допустим мы имеем некоторое референсное управление~$\bar u=\{\bar u^k\}_{k=1}^{N}$ и соответствующую ему референсную траекторию~$\bar x=\{\bar x^k\}_{k=1}^{N+1}$.
    Введем обозначения:
    $$
        f_x^k = \left.\frac{\partial f}{\partial x}\right|_{(\bar x^k, \bar u^k)},
        \quad
        f_u^k = \left.\frac{\partial f}{\partial u}\right|_{(\bar x^k, \bar u^k)},
    $$

    $$
        q^k = q(\bar x^k),
        \quad
        q_x^k = \left.\frac{\partial q}{\partial x}\right|_{\bar x^k},
        \quad
        q_{xx}^k = \left.\frac{\partial^2 q}{\partial x^2}\right|_{\bar x^k},
    $$

    $$
        r^k = r(\bar u^k),
        \quad
        r_x^k = \left.\frac{\partial r}{\partial u}\right|_{\bar u^k},
        \quad
        r_{xx}^k = \left.\frac{\partial^2 r}{\partial u^2}\right|_{\bar u^k}.
    $$

    Тогда, линеаризуя вдоль референсной траектории задачу Коши~\eqref{eq:discrete-system} и строя квадратичную аппроксимацию вдоль той же траектории функционала качества~\eqref{eq:discrete-cost}, получаем следующую задачу:
    \begin{equation}\label{eq:ref-system}
        \left\{\begin{aligned}
            &\delta x^{k+1} = f_x^k \delta x + f_u^k \delta u, \; k=\overline{1,N}, \\
            &\delta x^1 = 0.
        \end{aligned}\right.
    \end{equation}
    \begin{multline}\label{eq:ref-cost}
        J = q^{N+1} + q_x^{N+1}\tilde x^{N+1} + \frac{1}{2}\langle \tilde x^{N+1}, q_{xx}^{N+1}\tilde x^{N+1} \rangle
        + \\ +
        \sum_{k=1}^{N}\left[ q^{k} + q_x^{k}\tilde x^{k} + \frac{1}{2}\langle \tilde x^{k}, q_{xx}^{k}\tilde x^{k} \rangle \right]
        + \\ +
        \sum_{k=1}^{N}\left[ r^{k} + r_u^{k}\tilde u^{k} + \frac{1}{2}\langle \tilde u^{k}, r_{uu}^{k}\tilde u^{k} \rangle \right],
    \end{multline}
    где
    \begin{equation*}
        \tilde x^k = \bar x^k + \delta x^k, \qquad \tilde u^k = \bar u^k + \delta u^k.
    \end{equation*}

    Построим гамильтониан для задачи~\eqref{eq:ref-system}-\eqref{eq:ref-cost}:
    \begin{multline}\label{eq:hamiltonian}
        H_k = q^{k} + q_x^{k}\tilde x^{k} + \frac{1}{2}\langle \tilde x^{k}, q_{xx}^{k}\tilde x^{k} \rangle
        +\\+
        r^{k} + r_u^{k}\tilde u^{k} + \frac{1}{2}\langle \tilde u^{k}, r_{uu}^{k}\tilde u^{k} \rangle
        +\\+
        (\lambda^{k+1})\T (f_x^k \delta x^k + f_u^k \delta u^k),
    \end{multline}
    где $\lambda^{k+1}$~--- мультипликаторы Лагранжа.

    Оптимальное управление $\delta u$ должно удовлетворять необходимому условию
    \begin{equation*}
        \left.\frac{\partial H_k}{\partial (\delta u^k)}\right|_{\delta u^k = \delta u^{k*}}
        \!\!\!\!\!\!=
        r_u^k + r_{uu}^k(\bar u^k + \delta u^{k*}) + (f_u^k)\T\lambda^{k+1} = 0,
    \end{equation*}
    что дает следующее выражение для поправки:
    \begin{equation}\label{eq:delta-u}
        \delta u^{k*} = - (r_{uu}^k)^{-1}[(f_u^k)\T\lambda^{k+1} + r_u^k] - \bar u^k.
    \end{equation}
    При этом имеет силу сопряженная задача:
    \begin{equation}\label{eq:lambda-system}
        \left\{\begin{aligned}
            \lambda^k = (f_x^k)\T \lambda^{k+1} + q_x^k + q_{xx}^k(\bar x^k + \delta x^k)
            \\
            \lambda^{N+1} = q_x^{N+1} + q_{xx}^{N+1}(\bar x^{N+1} + \delta x^{N+1})
            .
        \end{aligned}\right.
    \end{equation}
    Из \eqref{eq:delta-u} и \eqref{eq:lambda-system} вытекает
    \begin{equation}\label{eq:delta-x-lambda-system}
        \begin{pmatrix}
            \delta x^{k+1}
            \\
            \lambda^{k}
        \end{pmatrix}
        =
        \underbrace{\begin{pmatrix}
            & f_x^k & -f^k_u (r^k_{uu})^{-1} (f^k_u)\T &
            \\
            & q^k_{xx} & (f^k_x)\T &
        \end{pmatrix}}_{\Phi^k}
        \begin{pmatrix}
            \delta x^k
            \\
            \lambda^{k+1}
        \end{pmatrix}
        +
        \underbrace{\begin{pmatrix}
            -f^k_u (r^k_{uu})^{-1} r^k_u
            \\
            q^k_x
        \end{pmatrix}}_{\Gamma^k}.
    \end{equation}

    \begin{theorem}
        Оптимальная поправка $\delta u$ для задачи~\eqref{eq:ref-system}-\eqref{eq:ref-cost} вычисляется как
        \begin{equation}\label{eq:method-optimal-control-result}
            \delta u^k = - (r^k_{uu} + (f^k_u)\T S_{k+1} f^k_u)^{-1} ((f^k_u)\T S_{k+1} f^k_u \delta x + v^{k+1} + r^k_u),
        \end{equation}
        где $S_k$ и $v^{k}$ высчитываются в обратном времени как
        \begin{equation}\label{eq:method-s-v-result}
            \begin{aligned}
                S_k &= \Phi^k_{21} + \Phi^k_{22} S_{k+1} (I - \Phi^k_{12}S_{k+1})^{-1} \Phi^k_{11},
                \\
                v^k &= \Phi^k_{22} S_{k+1} ( I - \Phi^k_{12} S_{k+1})^{-1} (\Phi^k_{12} v^{k+1} + \Gamma^k_1) + \Phi^k_{22} v^{k+1} + \Gamma^k_2
            \end{aligned}
        \end{equation}
        с граничными условиями
        \begin{equation}\label{eq:method-s-v-boundary-result}
            \begin{aligned}
                S_{N+1} &= q^{N+1}_{xx},
                \\
                v^{N+1} &= q^{N+1}_{x} + q^{N+1}_{xx}\bar x^{N+1}.
            \end{aligned}
        \end{equation}
    \end{theorem}
    
    \begin{proof}
        Предположим, что мультипликаторы $\lambda$ имеют следующую аффинную форму относительно фазовой переменной $\delta x$
        \begin{equation}\label{eq:lambda-affin}
            \lambda^{k} = S_k \delta x^k + v^k
        \end{equation}
        Тогда из граничного условия \eqref{eq:lambda-system} вытекает граничное условие на $S_k$, $v^k$ \eqref{eq:method-s-v-boundary-result}:
        \begin{equation*}
            \begin{gathered}
                \lambda^{N+1} = q_x^{N+1} + q_{xx}^{N+1}\left(\bar x^{N+1} + \delta x^{N+1}\right)
                \\ \Downarrow \\
                S_{N+1} = q^{N+1}_{xx}, v^{N+1} = q^{N+1}_{x} + q^{N+1}_{xx}\bar x^{N+1}.
            \end{gathered}
        \end{equation*}
        Теперь подставим \eqref{eq:lambda-affin} в выражение \eqref{eq:delta-x-lambda-system} для $\delta x^{k+1}$:
        \begin{equation*}
            \delta x^{k+1} = \Phi^k_{11} \delta x^k + \Phi^k_{12} (S_{k+1}\delta x^{k+1} + v^{k+1}) + \Gamma^k_{1}.
        \end{equation*}
        Получаем
        \begin{equation*}
                \delta x^{k+1}
            =
                \left(
                    \underbrace{
                        I - \Phi^k_{12} S_{k+1}
                    }_{K_k}
                \right)^{-1}
                \!\!\!\!\!\left(
                    \Phi^k_{11} \delta x^k + \Phi^k_{12} v^{k+1} + \Gamma^k_{1}
                \right).
        \end{equation*}
        Подставим получившееся выражение в \eqref{eq:delta-x-lambda-system} для $\lambda^k$:
        \begin{multline*}
                \lambda^{k} = S_k \delta x^k + v^k
            =
                \Phi^k_{21} \delta x^k + \Phi^k_{22} \left( S_{k+1} \delta x^{k+1} + v^{k+1} \right) + \Gamma^k_{2}
            =\\=
                    \Phi^k_{21} \delta x^k
                +
                    \Phi^k_{22}
                    \left(
                            S_{k+1} K_k^{-1} (\Phi^k_{11} \delta x^k + \Phi^k_{12} v^{k+1} + \Gamma^k_{1})
                        +
                            v^{k+1} 
                    \right)
                +
                    \Gamma^k_{2}.
        \end{multline*}
        Таким образом получаем искомые соотношения \eqref{eq:method-s-v-result}:
        \begin{equation*}
            \begin{aligned}
                S_{k} &= \Phi^k_{21} + \Phi^k_{22} S_{k+1} K_k^{-1} \Phi^k_{11},
                \\
                v^{k} &= \Phi^k_{22} (S_{k+1} K_k^{-1} ( \Phi^k_{12} v^{k+1} + \Gamma^k_{1} ) + v^{k+1} ) + \Gamma^k_{2}.
            \end{aligned}
        \end{equation*}
        Итоговая формула для оптимальной поправки \eqref{eq:method-optimal-control-result} получается прямой подстановкой получившихся соотношений в выражение \eqref{eq:delta-u}.

    \end{proof}
    
    \begin{remark}
        Полученная теорема требует существование обратных матриц для $K_k, k=\overline{1,N}$.
        При этом для нелинейных систем данное условие может не выполняться.
        Чтобы метод продолжал работать, предлагается в случае нулевого определителя $\textnormal{det}\,K_k = 0$, заменять в формулах \eqref{eq:method-optimal-control-result}, \eqref{eq:method-s-v-result} матрицу $K_k$ на регуляризованную
        \begin{equation}
            \mathcal{K}_k = K_k + \mu I.
        \end{equation}
    \end{remark}

    \ifSubfilesClassLoaded{
        \nocite{*}
        \clearpage
        \bibliographystyle{plain}
        \bibliography{../../refs}
    }{}
\end{document}