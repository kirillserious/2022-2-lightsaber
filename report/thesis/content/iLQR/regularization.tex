\documentclass[../../doc.tex]{subfiles}

\begin{document}
    \subsection{Регуляризация оптимальной поправки}

    Согласно Теореме 1 оптимальная поправка имеет следующую аффинную форму
    \begin{equation*}
        \delta u^{k*} = L_k \delta x^k + d^k,
    \end{equation*}
    где $L_k$~--- коэффициент управления с обратной связью, $d^k$~--- коэффициент управления без обратной связи, возникающий по причине того, что мы имеем дело с отклонениями от заданного состояния.

    Данная форма не налагает никаких ограничений на поправку $\delta u$.
    На практике это означает,
        что на начальных итерациях алгоритма,
        когда референсная траектория далека от оптимальной,
        поправка зачастую выводит систему за область действия аппроксимации.
    Визуально это выражается в том, что на каждой итерации алгоритм выдаёт некоторую случайную траекторию и в конечном итоге не сходится к оптимальной траектории.
    Чтобы избежать такого эффекта, необходимо регуляризовать коэффициент управления без обратной связи $d^k$:
    \begin{equation*}
        \delta u^{k*}(\eta) = L_k \delta x^k + \eta d^k.
    \end{equation*}

    Теперь остается ответить на вопрос, как выбрать подходящий коэффициент регуляризации~$\eta$.
    Это можно сделать двумя способами:
    \begin{enumerate}
        \item Дополнительно поточечно ограничить область допустимых управлений
        $$
            u^k \in \mathcal{U}^k.
        $$
        В таком случае можно выбирать коэффициент из соотношения
        $$
            \eta
            =
            \begin{cases}
                1, & \mbox{при } \bar u^k + \delta u^{k*} \in \mathcal{U}^k, k = \overline{1,N},
                \\
                \textnormal{min} \left\{ \eta \,|\, \bar u^k + \delta u^{k*}(\eta) \in \partial \mathcal{U}^k \right\}, & \mbox{иначе}.
            \end{cases}
        $$
        Такой способ предполагает всего один дополнительный проход алгоритма для поиска минимума,
        однако накладывает ограничения, не предусмотренные исходной задачей,
        и не даёт гарантии на то, что в среднем отклонение в конечном счёте не накопится.

        \item Использовать ожидаемое отклонение от функции цены
        \begin{equation}\label{eq:ilqr-regularization}
            \xi_1
            \leqslant
            \frac{J(\bar u) - J(\bar u + \delta u^{*}(\eta))}{J_{\delta}(0) - J_{\delta}(\delta u^{*}(\eta))}
            \leqslant
            \xi_2.
        \end{equation}
        Данный способ не накладывает дополнительных ограничений.
        При этом для нахождения правильного коэффициента может потребоваться нес\-колько итераций.
        Изначально коэффициент $\eta$ инициализируется единицей, затем считается оптимальная поправка, и если условие \eqref{eq:ilqr-regularization} не выполнено, то коэффициент уменьшается $\eta \gets \gamma \eta$, где $\gamma < 1$.
    \end{enumerate}

    \begin{remark}
        В случае задачи с заданными ограничениями на управление необходимо комбинировать оба рассмотренных условия.
        Для численного решения нашей задачи было использовано второе.
    \end{remark}

    \ifSubfilesClassLoaded{
        \nocite{*}
        \clearpage
        \bibliographystyle{plain}
        \bibliography{../../refs}
    }{}
\end{document}