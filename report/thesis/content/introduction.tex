\documentclass[../doc.tex]{subfiles}
\graphicspath{{\subfix{../img}}}
\begin{document}
    Работа посвящена построению оптимальных траекторий движения в модели бионической руки человека,
        держащего предмет.
    Целью работы является построение эффективного вычислительного метода
        для управления сложными биомеханическими системами.
    Для разработки и тестирования построенного метода
        была предложена соответствующая математическая модель.
    
    Исследования в области биологического движения имеют огромное практическое значение:
        они позволяют частично восстановить двигательную фун\-кцию у людей
        с ограниченными возможностями,
        чем существенно улучшить качество их жизни.
    Последние технические достижения в области роботизированных протезов
        и функциональной электронной стимуляции парализованных мышц
        позволяют начать внедрение данной теории.
    Более того, сложные бионические устройства становятся бесполезными без соответствующего знания
        о грамотном управлении ими.

    В работе предложена математическая модель руки человека, держащего предмет,
        как планарного трёхсекционного математического маятника.
    Выведена динамика данной физической системы.
    Для возможности анализа поведения системы методами оптимального управления
        постулирован принцип оптимальности
        и рассмотрены представленные в литературе формализации
        оптимизационного энергетического критерия.
    
    Использование методов оптимального управления способно привести фундаментальным открытиям в области
        биологической моторики: от описания свойств функций отдельных мышц, до исследования контроля
        мышц нервной системой при выполнении целевых задач, ---
        поскольку данные методы напрямую работают с причинами движений, выраженными в форме оптимизационных критериев.

    По результатам математического моделирования была поставлена задача оптимального целевого управления нелинейной системой в непрерывной и дискретной формах.
    В качестве управления выбрано изменение крутящего момента каждого из сочленений.
    В работе полагается отсутствие ограничений на управление и известное полное фазовое состояние системы.

    Для решения задачи в дискретной постановке были рассмотрены известные базовые методы решения задачи
        оптимального управления нелинейными системами.
    В качестве основного метода в данной работе применяется итеративный метод,
        предполагающий последовательное построение серии линейно-квадратичных регуляторов
        для системы и функционала качества, аппроксимированных вдоль заданной траектории.
    
    При разработке метода особое внимание было уделено аспектам, не достаточно подробно
        изложенным в литературе: способу регуляризации оптимальной поправки и способу
        построения начальной референсной траектории без опоры на мнение эксперта в предметной
        области.

    Полученное в результате программное решение, реализующее предложенный метод, 
        было применено для рассмотрения конкретных постановок задачи.
    Большая часть постановок --- классические задачи биомеханического движения: задача целевого положения схвата, задача обхода препятствия.
    Они служат возможности сравнения результатов работы предложенного метода с имеющимися в литературе.

    \ifSubfilesClassLoaded{
        \nocite{*}
        \clearpage
        \bibliographystyle{plain}
        \bibliography{../../refs}
    }{}
\end{document}