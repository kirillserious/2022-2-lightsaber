\section{Итеративный метод cинтеза оптимального управления}

\subsection{Общая идея метода}

Есть хорошо разработанная теория для решения интегральных линейно-квадратичных задач.
Большинство работ, посвященных управлению нелинейными системами, предлагают линеаризацию задачи с потерей физического смысла управления (давайте минимизировать то, что мы умеем минимизировать).
Поэтому далее предложен метод, который решает эту проблему.
Общая идея метода схожа с идеей метод дифференциального динамического программирования.

Метод итеративный.
\begin{enumerate}
    \item Предположим, что на $k$-ой итерации мы имеем некоторое \textit{референсное} управление $\bar{u}^k$ и соответствующую ему референсную траекторию $\bar{x}^k$.
    \item Линеаризуем систему и функционал качества в окрестности референсной траектории и построим поправку $\delta u$ для референсного управления $\bar{u}^{k+1} = \bar{u}^k + \delta u$.
    \item Используем поправленное управление в качестве референсного на следующей итерации алгоритма.
\end{enumerate}
Критерий остановки алгоритма, если $|J(u^k) - J(u^{k-1})| < \varepsilon$ для некоторого заданного наперед $\varepsilon > 0$.

\subsection{Применение метода для классических задач}

Тут будут примеры с картинками, как данный метод применяется для задач:
\begin{itemize}
    \item Перехода в целевое состояние
    \item Перехода в целевое положение схвата
    \item Обход препятствия
\end{itemize}