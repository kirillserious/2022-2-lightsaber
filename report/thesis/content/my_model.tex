\section{Математическое моделирование}

\subsection{Дискретизация задачи}

{\color{red} Тут уже чувствуется, что буквы начали повторяться.}

Для удобства дальнейших рассуждений дискретизируем задачу~\eqref{eq:kinematic}-\eqref{eq:cauchy}-\eqref{eq:continuos-cost} по времени $t_0 \leqslant t \leqslant t_1$.
Для этого введем равномерную сетку с шагом~$\Delta t$:
$$
    \{ t_i \}_{i=0}^{N+1}, \quad t_0 = t_0, \quad t_N = t_1, \quad t_{i+1} - t_{i} = \Delta t.
$$
Тогда, сузив класс допустимых управлений до кусочно-постоянных, получаем дискретный вариант рассматриваемой задачи Коши~\eqref{eq:kinematic}-\eqref{eq:cauchy}:
\begin{equation}\label{eq:discrete-system}
    \left\{\begin{aligned}
        &x^{k+1} = f(x^k, u^k), \\
        &x^{0} = x^{0},
    \end{aligned}\right.
\end{equation}
где $f(x^k, u^k) = \Delta t (A(x^k) + B u^k) + x^k$.

При этом функционал \eqref{eq:continuos-cost} для дискретной задачи приобретет вид
\begin{equation}\label{eq:discrete-cost}
    J = q^{N+1}(x^{N+1}) + w_1 \sum_{i=0}^{N} \Delta t q(x^N) + w_2 \sum_{i=0}^{N} \Delta t r(u^N).
\end{equation}
