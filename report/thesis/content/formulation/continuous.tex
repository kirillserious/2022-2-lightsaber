\documentclass[../../doc.tex]{subfiles}
\graphicspath{{\subfix{../../img}}}
\begin{document}
    \subsection{Непрерывная постановка задачи}
    
    Поставим задачу целевого управления для модели, построенной в Разделе~1.
    Для этого рассмотрим расширенное фазовое пространство с состоянием
    \begin{equation*}
        x = [\theta\;\dot\theta\;\tau]\T \in \mathbb{R}^9.
    \end{equation*}
    Тогда уравнение динамики системы~\eqref{eq:dynamic} можно переписать в виде системы однородных дифференциальных уравнений
    \begin{equation}\label{eq:kinematic}
        \dot x = A(x) + Bu,
    \end{equation}
    где 
    $$
        A(x) = \begin{bmatrix}
            x_2 \\
            M^{-1}(x_1)(x_3 - L(x_1, x_2)) \\
            O
        \end{bmatrix}
        ,\quad
        B = \begin{bmatrix}
            O \\
            O \\
            I
        \end{bmatrix}.
    $$
    Считаем, что для данной системы поставлена задача~Коши, то есть нам известно начальное состояние системы
    \begin{equation}\label{eq:cauchy}
        x(t_0) = x^{\textnormal{start}}.
    \end{equation}

    \begin{remark}
        Отметим, что для выполнения достаточных условий существования и единственности решения Каратеодори для задачи Коши~\eqref{eq:kinematic}-\eqref{eq:cauchy} управление~$u$ достаточно брать из класса измеримых на рассматриваемом отрезке $t_{\textnormal{start}}\leqslant t \leqslant t_{\textnormal{final}}$ функций.
    \end{remark}

    Для задачи Коши~\eqref{eq:kinematic}-\eqref{eq:cauchy} поставим задачу поиска управления $u \in U[t_{\textnormal{start}},t_{\textnormal{final}}]$, минимизирующего функционал вида:
    \begin{equation}\label{eq:continuos-cost}
        J = q^{\textnormal{final}}(x(t_{\textnormal{final}})) + w_1 \!\! \uint\limits_{t_{\textnormal{start}}}^{t_{\textnormal{final}}} \!\! q(x(t))\,dt + w_2 \!\! \uint\limits_{t_{\textnormal{start}}}^{t_{\textnormal{final}}} \!\! r(u(t))\,dt,
    \end{equation}
    где $q^{\textnormal{final}}$, $q$ отвечают за терминальное и фазовые ограничения соответственно и выбираются в зависимости от конкретной постановки задачи, а $r$ отвечает за энергетические затраты и в соответствии с \eqref{eq:energy-cost} равна:
    $$
        r(u) = \|u\|^2,
    $$
    а $w_1$, $w_2$~--- веса соответствующих критериев для данной многокритериальной задачи.

    Для дальнейших рассуждений потребуем, чтобы функции $q^{\textnormal{final}}$, $q$ были дважды непрерывно дифференцируемыми.
    Полученные из модели функции $A$ и $r$ заведомо удовлетворяют этому требованию.

    \begin{remark}
        Учёт фазовых ограничений в интегральной части функционала качества $J$, представленный в работе,
        позволяет лишь приближенно уписать условия вида
        $$
            g_i(x) \leqslant 0,
        $$
        которые часто встречаются в задачах, например, для обхода препятствия.
        Для этого функция $q$ выбирается таким образом, чтобы штрафовать за приближение траектории к препятствию.
        Для строго формального решения задачи с подобным условием,
        необходимо пользоваться методами расширенного лангранжиана~\cite{birgin2009},
        которые предполагают решение серии задач типа~\eqref{eq:kinematic}-\eqref{eq:cauchy}-\eqref{eq:continuos-cost}.
        Это приводит к существенному ухудшению асимптотики алгоритмов и тем самым увеличению времени работы программного решения.
    \end{remark}

    \ifSubfilesClassLoaded{
        \nocite{*}
        \clearpage
        \bibliographystyle{plain}
        \bibliography{../../refs}
    }{}
\end{document}