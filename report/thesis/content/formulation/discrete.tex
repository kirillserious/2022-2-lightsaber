\documentclass[../../doc.tex]{subfiles}
\graphicspath{{\subfix{../../img}}}
\begin{document}
    \subsection{Дискретизация задачи}

    Для удобства дальнейших рассуждений дискретизируем задачу~\eqref{eq:kinematic}-\eqref{eq:cauchy}-\eqref{eq:continuos-cost} по времени $t_{\textnormal{start}} \leqslant t \leqslant t_{\textnormal{final}}$.
    Для этого введем равномерную сетку с шагом~$\Delta t$:
    $$
        \{ t_i \}_{i=1}^{N+1}, \quad t_1 = t_{\textnormal{start}}, \quad t_{N+1} = t_{\textnormal{final}}, \quad t_{i+1} - t_{i} = \Delta t.
    $$
    Тогда, сузив класс допустимых управлений до кусочно-постоянных, получаем дискретный вариант рассматриваемой задачи Коши~\eqref{eq:kinematic}-\eqref{eq:cauchy}:
    \begin{equation}\label{eq:discrete-system}
        \begin{cases}
            x^{k+1} = f(x^k, u^k),\; k=\overline{1,N}, \\
            x^{1} = x^{\textnormal{start}},
        \end{cases}
    \end{equation}
    где
    \begin{equation*}
        f(x^k, u^k) = \Delta t \left(A(x^k) + B u^k \right) + x^k.
    \end{equation*}

    При этом функционал \eqref{eq:continuos-cost} для дискретной задачи приобретет вид
    \begin{equation}\label{eq:discrete-cost}
        J = q^{N+1}(x^{N+1}) + \sum_{k=1}^{N} q^k(x^k) + \sum_{k=1}^{N} r^k(u^k),
    \end{equation}
    где
    \begin{equation*}
        q^{N+1} = q^{\textnormal{final}},
        \qquad
        q^k = w_1 q \Delta t,
        \qquad
        r^k = w_2 r \Delta t
        \qquad
        k = \overline{1,N}.
    \end{equation*}

    \ifSubfilesClassLoaded{
        \nocite{*}
        \clearpage
        \bibliographystyle{plain}
        \bibliography{../../refs}
    }{}
\end{document}