\documentclass[../../doc.tex]{subfiles}
\graphicspath{{\subfix{../../img}}}
\begin{document}
    \subsection{Учёт энергетических затрат}

    Для моделирования биологического движения необходимо выяснить, какими принципами руководствуется мозг при выборе траектории для некоторого целевого движения.
    Существует бесконечное число возможных путей и профилей скорости для перемещения руки из одной точки в другую,
    и каждая траектория может быть достигнута несколькими возможными комбинациями углов между сочленениями.
    При этом нервная и моторно-двигательные системы человека для выбора одной конкретной траектории анализируют большой объем информации, поступающий от всех органов чувств.

    В силу того, что нервная система человека есть результат оптимизационных процессов:
    эволюции, адаптации к условиям среды, обучения,
    мы постулируем следующий биологический принцип оптимальности.
    
    \begin{assertion}[Биологический принцип оптимальности]
        Выбираемые нервной системой схемы движения являются оптимальными для поставленной задачи.
    \end{assertion}
    
    Применение данного принципа позволяет не только моделировать движения методами оптимального управления, но и анализировать их причины.

    В работе \cite{todorov2002} было показано, что оптимизации проводятся с целью уменьшения затрат энергии.
    Однако общего подхода к формализации энергетических затрат пока не выработано.
    Так, например, в работе~\cite{hogan1984} предлагается минимизировать \textit{рывок} схвата, то есть
    $$
        \uint\limits_{t_{\textnormal{start}}}^{t_{\textnormal{final}}}\left\|
            \frac{d^3e^3}{dt^3}
        \right\|^2\, dt \longrightarrow \textnormal{min},
    $$
    а в работе~\cite{uno1989}~--- изменение крутящего момента
    \begin{equation}\label{eq:energy-cost}
        \uint\limits_{t_{\textnormal{start}}}^{t_{\textnormal{final}}}\left\|
            \frac{d\tau}{dt}
        \right\|^2\, dt \longrightarrow \textnormal{min}.
    \end{equation}
    Причём существуют и другие менее популярные варианты, например, \cite{harris1998}.
    
    Мы будем использовать для формализации энергетических затрат выражение~\eqref{eq:energy-cost},
    поскольку данный критерий напрямую зависит от динамики руки и лучше согласуется с эмпирическими данными,
    чем модель минимального рывка~\cite{breteler2002}.
    
    \ifSubfilesClassLoaded{
        \nocite{*}
        \clearpage
        \bibliographystyle{plain}
        \bibliography{../../refs}
    }{}
\end{document}