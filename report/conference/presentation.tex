\documentclass[16pt]{beamer}

\usepackage[utf8]{inputenc}
\usepackage[main=russian, english]{babel}
\usepackage{amssymb}
\usetheme{Warsaw}

\title[Конференция <<Ломоносовские чтения 2023>>]
        {Математическое моделирование движения руки и поведенческих движений}
\author[К. Ю. Егоров]
        {студент 2 курса магистратуры К. Ю. Егоров\\
        научный руководитель --- к.ф-м.н., доцент И. В. Востриков}
\institute{Кафедра системного анализа\\ ВМК МГУ}
\date{\today}

% Добавляем в панель навигации номер страницы
\setbeamertemplate{navigation symbols}{
        \insertslidenavigationsymbol
        \insertframenavigationsymbol
        \insertsubsectionnavigationsymbol
        \insertsectionnavigationsymbol
        \insertdocnavigationsymbol
        \insertbackfindforwardnavigationsymbol
        \hspace{1em}
        \usebeamerfont{footline}
        \insertframenumber /\inserttotalframenumber
        %
}

\begin{document}
    \begin{frame}
        \titlepage
    \end{frame}

    \begin{frame}
        \frametitle{Зачем это всё}
        Lorem ipsum dolor sit amet, consectetur adipisicing elit, sed do eiusmod tempor incididunt ut labore et dolore magna aliqua.
    \end{frame}

    \begin{frame}
        \frametitle{Математическое моделирование}
        Метод Эйлера--Лагранжа
        $$
            \mathcal{L} = \Pi - K
            \;\Longrightarrow\;
            \tau_i
            =
              \frac{d}{dt}\left(\frac{\partial \mathcal{L}}{\partial \dot \theta_i}\right)
            - \frac{\partial \mathcal{L}}{\partial \theta_i}
        $$
        \begin{block}{Уравнение динамики}
            $$
                \tau = M(\theta)\ddot\theta + L(\theta, \dot\theta)
            $$
            \begin{itemize}
                \item $\tau_i$~--- момент силы, действующей на $i$-е сочленение
                \item $M(\theta)=M^{\mathrm{T}}(\theta)>0$~--- матрица инерции
                \item $L(\theta, \dot\theta)$~--- вектор центростремительных и корелисовых сил
            \end{itemize}
        \end{block}
    \end{frame}

    \begin{frame}
        \frametitle{Трудозатраты}
        Lorem ipsum dolor sit amet, consectetur adipisicing elit, sed do eiusmod tempor incididunt ut labore et dolore magna aliqua.
        \begin{block}{Формализация энергетических затрат}
            $$
                \mbox{Затраты}\, = \int\limits_{t_{start}}^{t_{final}}\dot\tau\,dt
            $$
        \end{block}
    \end{frame}

    \begin{frame}
        \frametitle{Задача достижимости}
        Введем $x = [\theta\;\dot\theta\;\dot \tau]^{\mathrm{T}}$, тогда
        \begin{align*}
            &\dot x = f(x)u + g(x)\\
            &x(t_{start}) = x^{start}
        \end{align*}
        Задача минимизации функционала:
        $$
            J = \langle x-x^{final}, Q^{final}(x-x^{final}) \rangle + \int\limits_{t_{start}}^{t_{final}}\langle u, Ru\rangle\,dt
        $$
    \end{frame}

    \begin{frame}
        \frametitle{Дискретизация}
        $$
            \dot x = f(x, u)\;\Longrightarrow\; x^{k+1} = \Delta t \cdot f(x^{k}, u^{k}) + x^{k}
        $$
    \end{frame}

    \begin{frame}{Референсная траектория}
        \begin{block}{Зачем}
            \begin{itemize}
                \item Есть разработанная теория для решения линейно-квадратичных задач
                \item Наша задача не такая
                \item Но мы можем её линеаризовать
                \item Но нужно знать, вокруг какой траектории это делать
            \end{itemize}
        \end{block}
        Построим референсную траекторию заменой задачи на линейную:
        $$
            v = Mu + L\;\Longrightarrow\;
        $$
    \end{frame}

    \begin{frame}{Референсная траектория}
        Метод динамического программирования даёт решение данной задачи:
        $$
            Here a formulae be provided
        $$
        \begin{block}{Почему она хороша}
            \begin{itemize}
                \item Она приводит нас к целевому положению
                \item Она минимизирует изменение углового ускорения, значит эта траектория возможна
                \item Но она не имеет никакого отношения к энергии
            \end{itemize}
        \end{block}
    \end{frame}

    \begin{frame}
        \frametitle{Будем улучшать}
        Вдоль референсной траектории $(u,x)$ линеаризуем систему:
        $$
            \delta x^{k+1} = A^{k}\delta x^k + B^k \delta u^k,\;\mbox{где}\; A^k = \left.\frac{\partial f}{\partial x}\right|_{(x,u)}
        $$
    \end{frame}

    \begin{frame}
        \frametitle{Дифференциальное динамическое программирование}
        Lorem ipsum dolor sit amet, consectetur adipisicing elit, sed do eiusmod tempor incididunt ut labore et dolore magna aliqua.
    \end{frame}
\end{document}