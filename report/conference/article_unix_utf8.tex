\begin{lmrarticle}%
    {Математическое моделирование движения руки и поведенческих движений}{Егоров~К.\,Ю., Востриков~И.\,В.}
    \TwoAuthor%
    {Егоров Кирилл Юлианович}%
        {Кафедра системного анализа}{kireku@gmail.ru}%
    {Востриков Иван Васильевич}%
        {Кафедра системного анализа}{ivan\_vostrikov@cs.msu.ru}
    
    Модель оптимального управления биологическим движением обеспечивает отправную точку для описания наблюдаемого повседневного поведения и может быть использована для конструирования интеллектуальных систем замены поврежденных частей тела.
    
    В работе методом Эйлера--Лагранжа~[1] построены уравнения движения планарной модели руки человека <<плечо-предплечье>> как бисекционного маятника:
    \begin{equation*}
        \tau = M(\theta)\ddot\theta + L(\theta, \dot\theta),
    \end{equation*}
    где $\theta$~--- вектор углов сочленений, $M(\theta)$~--- положительно определенная матрица инерции, $L(\theta, \dot\theta)$~--- вектор центростремительных и корелисовых сил, $\tau$~--- вектор крутящих моментов, которыми можно управлять.

    Для данной модели решается задача на достижение цели, в которой рука должна начать движение с некоторого исходного положения и двигаться к цели за указанный интервал времени~$T$.
    При этом движение должно происходить с минимальными затратами энергии.
    Есть веские основания полагать, что минимизация энергетических затрат имеет прямое отношение к нейронному контролю движения~[2].
    Затраты учитываются интегрально-квадратичным критерием, минимизирующий скорость изменения крутящего момента~[3]:
    \begin{equation*}
        V_{energy} = \mbox{$\frac{1}{2}\int_{0}^{T}$}\;\|\dot \tau\|^2\,dt.
    \end{equation*}

    Большинство современных моделей оптимальности имеют серьезное ограничение — они основаны на линейно-квадратичном гауссовом формализме~[4], в то время как в действительности биомеханические системы сильно нелинейны.
    Данное ограничение решается методом дифференциального динамического программирования~[5], позволяющего итеративно проводить линеаризацию системы вокруг референсной траектории с целью вычисления локально оптимального закона управления с обратной связью.

    Исходное референсное управление предлагается выбирать путем преобразования исходной динамики к линейной~[6], применяя подходы линейного управления.
    Затем градиентным методом~[7] итеративно строится поправка на референсное управление до достижения заданной точности $\varepsilon$. 
    
    Для данной задачи было реализовано программное обеспечение для поиска целевого управления и моделирования оптимальной траектории для заданных начальных данных.
    
    \begin{lmrreferences}
    \item Колюбин~С.\,А. Динамика робототехнических систем~// Учебное пособие. --- СПб.: Университет ИТМО, 2017.~--- 117 с.
    \item E. Todorov, M. Jordan. Optimal feedback control as a theory of motor coordination~// Nature Neuroscience, Vol.5, No.11, 1226-1235, 2002.
    \item Y. Uno, M. Kawato, R. Suzuki. Formation and control of optimal trajectory in human multijoint arm movement~--- minimum torque-change model~// Biological Cybernetics 61, 89-101, 1989.
    \item B.D.O. Anderson, J.B. Moore. Optimal Control: Linear Quadratic Methods~// Prentice Hall, Upper Saddle River, 1990.
    \item D. H. Jacobson. Differential dynamic programming methods for determining optimal control of non-linear systems~// University of London, 1967.
    \item E. Guechi, S. Bouzoualegh, Y. Zennir, S. Blažič. MPC Control and LQ Optimal Control of A Two-Link Robot Arm: A Comparative Study~// Machines 6, no. 3: 37, 2018.
    \item A. Babazadeh, N. Sadati. Optimal control of multiple-arm robotic systems using gradient method~// IEEE Conference on Robotics, Automation and Mechatronics, Singapore, pp. 312-317 vol.1, 2004.
    \end{lmrreferences}
    \end{lmrarticle}
    